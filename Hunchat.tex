% !BIB TS-program = biber

\documentclass[11pt]{article}
% \usepackage{lipsum}
\usepackage[margin=1in,includefoot]{geometry}
\usepackage[backend=biber, style=apa]{biblatex}
\usepackage{setspace}
\usepackage{multirow}
%\usepackage{mathptmx}

\addbibresource{/Users/tiagomarques/Desktop/hunchat/hunchat.bib}
 \onehalfspacing
 \setlength{\parskip}{1em}

\begin {document}

%Title page
\begin{titlepage}
\title{Integrated Marketing Communication Plan for Hunchat}
\author{Tiago Marques}
\maketitle
\end{titlepage}

\pagenumbering{roman}

%Summary
\section*{Summary}
\addcontentsline{toc}{section}{\numberline{}Summary}
This dissertation is an Integrated Marketing Communication (IMC) plan for the newcomer social media app called Hunchat.
  \par
 \textbf{Keywords} IMC, Hunchat, Social Media
\cleardoublepage

\thispagestyle{empty}

%Contents
\tableofcontents
\cleardoublepage 
\listoffigures
\listoftables 
\thispagestyle{empty}
\cleardoublepage 
\setcounter{page}{1}
\pagenumbering{arabic}
 
%Introduction
\section{Introdutction}\label{intro}

\newpage	
%Literature Review
\section{Literature Review}\label{review}
\subsection{A Brief History of Integrated Marketing Communications}\label{IMC}
The last two decades of the twentieth century have brought several innovations on the way companies were managed. The exacerbation of capitalism, deregulation of markets in the western nations due to the increasing prevalence of economic liberalism have created a highly competitive environment that required more creativity on the strategic side of business managers in order to stay relevant \cite{}. Integrated Marketing Communication (IMC) is one of those innovations whose beginnings go back to the eighties, but whose mainstream popularity only came in the nineties and has ever since influenced over marketing communication decisions. In a world where advertising was the center of all marketing practice, IMC plans were surely a competitive advantage for its users. Its results came to break the ruling paradigm - in 1985 the great majority  of companies' marketing budget (75\%) was spent on advertising. That number decreased to 25\% in 2005 - meaning along those 20 years resources were gradually more widely spread across all communication channels (\cite{holm}).  

Ever since multiple definitions for the term have surfaced. Some simplifying it to the mere management and control of all market communications, others defending it also implies ensuring the brand positioning, personality and message are all congruent, and some even adding its need to be the most efficient (the best use of resources), economical (at minimum cost), and effective (achieving maximum results) strategy (\cite{smith}). The main ideia behind it is - as the name suggests - a consistent delivery of a chosen message across all market communications or as  \citeauthor{kitchen} (\citeyear{kitchen}) argued - the goal is to create a 'one-voice' brand phenomenon (p. 19) that can occur at one or more of the different levels of integration: (1) vertical - the assimilation of marketing goals and corporate goals; (2) horizontal - marketing communications aligned with management functions; (3) marketing mix - having the mix (price, product and place) following promotion decisions; (4) communications mix - tools guiding customers consistently; (5) creative design - executed accordingly to the product positioning; (6) internal and external - assure all departments and employed agencies work based on an equal plan and strategy; and (7) financial - the use of the budget as most effective and efficient as possible (\cite{smith}). 

The beginning of the second millennium reinvigorated IMC with a more strategical approach instead of the widespread tactical one. Instead of focusing on short-term goals, companies began to grasp the perks of the competitive advantage given by thinking long-term. Organisations were now more focused on what was their identity, profile and overall image and made sure the scope of organisation's activities reflected what their owners wanted them to be like (\cite{holm}). Ever since the development of a wide new range of technological advancements - mainly drawn by the widespread use of the internet - gave rise to brand new way of communicating which impacted almost every aspect of human civilization meant the entire marketing communications system was no longer tilted in favour of the seller (\cite{kliatchko}). The fabric of communication was changing and the number of new media growing - computers, smartphones, tablets, search engines, social media, online forums...  Just as three notes can compose some songs but twelve notes can make compose any song, the rise of new media means more and different opportunities to communicate. The emphasis shifted from unilateral mass communication to bilateral targeted communication. What do such drastic changes on the mediatic landscape mean for IMC? Is integrated communication still important in this new world?

 \subsection{IMC in the Digital Age}\label{digitalage}
 
What once was an advantage may have become a disadvantage. The rise of a new diverse collection of IMC options created a lot of new problems on the integration of messages, furthering the misalignment between strategy and tactics (\cite{holm}.) The huge impact of the digital revolution on culture and society has manifested mainly by the way we communicate. The creation of social media completely broke the ruling paradigms of mass communication. No one could ever predict the groundbreaking network based media that is now so ubiquitous (or at least at this extent), but awareness of the power of technological advancements were quite common on researching IMC. Some expected the development of a media anarchy (\cite{solomon}), others that it would need to account user generated content (\cite{ananda}). Nevertheless the core model regarding the more tactical and strategic issues (such as \citeauthor{schultz} in \citeyear{schultz} or \citeauthor{duncan} in \citeyear{duncan}) although designed with traditional media in mind still applies to modern communication channels such as social media for their conceptual frameworks do not raise any significant barriers in its implementation. That is not the case for much of the traditional IMC models being developed when most media interactions were outsourced since they do not consider the implications of message delivery in so many, volatile, and widely specific channels. As \citeauthor{mcluhan} famously pointed out "the medium is the message" (\citeyear{mcluhan}), and with so many different media each with its unique specificity (history, context, surrounding communities, mediatic traits...) it is quite challenging (if not impossible) to encompass the same core message across all of them. The problem is no longer only which message to tell, but also which medium suits it best. It isn't only this larger amount of touch points that's threatening, but also their nature. Direct messaging and user generated content like comments, publications, stories... shift or at least balance the control over the message from the organisation towards the consumers, only aggravating the already difficult task of communicating consistently. Social Media also introduced a blurring of marketing functions into one - sales, promotion and services are now mushed together and sometimes indistinguishable from each other (\cite{valos}).  

While raising all of these implementation problems for IMC, social media has also brought an infinite amount of new and different opportunities to communicate. The real time feedback from consumers, the ability to better monitor the performance of campaigns and its impacts, the larger amount of analytics and insights collected from consumers can all strongly contribute to a better and more informed IMC decision making. It also forms a much more efficient, effective and economical way of communicating since it's delivered only at the targeted people - highly reducing the churn rate - for the desired amount of time, at the best context with as much specificity wanted for a small parcel of the cost of traditional media. The engagement from consumers provides an ongoing interaction, giving more agility and possibilities of integration. If there's a party winning for its presence on social media it is the advertisers. The abandonment of a sporadic one way communication should be seen not as a threat to IMC, but as an improvement on its ability to form a continuous, transparent and free dialogue between customers and organisations (\cite{ananda}). 
 
It is also important to point out that modern communication on social media is usually mediated by intelligent black box algorithms by which many of market decisions are made - whether it is by high frequency stock trading (\cite{hft}) or personal feed curation (\cite{algo}). What this means for IMC is that it should account for these new artificial 'readers' of content as to achieve the maximum exposure, considering its possible effects on all stakeholders without jeopardising the original message.  

 \subsection{The Importance of Communication for Social Networks}\label{SN}

A Social Network is not only a medium of communication but it also needs to be communicated itself. The market of social media apps is quite large - there are fifty thousand new competitors every month (on the app store) all trying to be the next facebook (SOURCE) - and so standing out from the huge crowd is a challenge and communication is a crucial help.
 
Analysing the narratives that big players in the industry have used to rise to the top there is a clear commonality: counter-culture. That is not unique for social networks but for most tech companies whose beginnings were viewed as the ultimate hope for social change. The fact that digital technology seemed free of human fragilities, it promised to solve most human made societal problems. Even though the hopeful prospects for tech might have not fully realised, all major social networks still have each broken some ruling paradigm - instagram revolutionised the way pictures are shared online, facebook changed the way we share our personal information on the web, whatsapp took online chatting and made it personal - and most made sure to communicate that revolution (\cite{evil}).
 
What really makes communicating a social network so different from any other business is the importance of reaching the right people. The product sold is not only the tool itself but mainly the people who use it (hence the name 'social network') - one does choose which social media app to use mostly based on its participants. That not only implies a necessity of a minimum amount of people using the product, but also a concern for which people exactly. A common strategy to solve the participation problem is to ensure exclusivity. Facebook has achieved it by exclusively accepting new users who were directly invited by those already using it (imitating universities' social clubs) (\cite{zucked}). This strategy not only ensures that new users already have other users to connect with (which is the main goal of a social network), but can also serve as a form of buzz marketing - one must always wonder what's happening behind closed gates. The success of clubhouse has just proven how effective opting for this path can still be.
 
After reaching a solid base of users the next big challenge in communicating a social network is to stay relevant. There are plenty of social networks who manage the difficult task of reaching a large amount os users but then lack in means to keep them interested. 

For the ones who do overcome all these hurdles it then arises the bigger struggle. As it has been shown by almost every social network achieving this long-term widespread popularity, the big tech monopolies always want a piece of the pie and will try to acquire it. That was the case for WhatsApp, Instagram (bought by Facebook), Youtube (bought by Google), LinkedIn, Discord (bought by Microsoft), Tumblr (bought by Yahoo) or Twitch (bought by Amazon). The ones who are brave enough to resist and persist on continuing on their own usually end up being copied and then have the difficult task of competing with their own product with the added hassle of the competitor's ownership of a larger, more engaged, and already fully developed network of people. That was the case for TikTok, Snapchat (copied by Facebook) . Then there are the small few cases who despite this arduous environment still manage to retain its network. That is the case of Twitter and Reddit.

All this raises three questions: (1) How can a social network differentiate from the big ones as a newcomer?; (2) How can a social network stay relevant after reaching a large audience?; (3) How can a social network stand out even when larger competitors are able to use the tools that differentiated them in the first place?; All questions in which communication can be a crucial part of the solution. Therefore such hazardous challenges require the use an IMC conceptual framework capable of addressing all of them.

 \subsection{IMC Frameworks}\label{fw}
 
 The almost forty long years of IMC existence made it gradually evolve to ever more complete frameworks. We have come a long way since the view of \citeauthor{nowak} (\citeyear{nowak}) passing through the \citeauthor{kliatchko}'s people based view framework (\citeyear{kliatchko}) and \citeauthor{valos}  (\citeyear{valos}) already devised a decision making framework that integrates social media in the IMC process. 


%Conceptual Framework
\section{Conceptual Framework}\label{conceptualfw}

The superiority \citeauthor{clow}'s framework for IMC is clear - not only for its complete and robust body of contextual analysis and practical application tools, but also for its strategic versatility (and more abstract nature) enabling an easy integration of social media. Even though this framework excels in assuring an overall communication integration, it lacks in considering how the use of brand elements in accordance with the key message might be crucial to a full integrated marketing communication (and specifically on this context as seen on \ref{SN}). For that reason I will encompass \citeauthor{kliatchko}'s brand and consumer auditing phases from the framework he proposed in \citeyear{kliatchko}. The conceptual framework for the IMC plan for Hunchat will then be as represented in table \ref{table:concept}. 

\begin{table}  [htb]
\centering
\caption{Conceptual Framework}
\label{table:concept}
\begin{tabular}{lll}
Author             &  Plan Phase(s)   & Data Collection  \\ 
\hline
\cite{kliatchko} & \begin{tabular}[c]{@{}l@{}}(1) Consumer Auditing;\\ (2) Brand Auditing\end{tabular}                                                                                                                                                                                         & \multirow{2}{*}{\begin{tabular}[c]{@{}l@{}}Waiting List Survey (\ref{qwait})\\ Interview with founder (\ref{ceo}) \\ Survey (\ref{qtarget}) \end{tabular}} \\ \cline{1-2}
 \cite{clow}  & \begin{tabular}[c]{@{}l@{}}(3) Internal Analysis; \\ (4) External Analysis; \\ (5) SWOT Analysis; \\ (6) Goals; \\ (7) Strategy; \\ (8) Messages; \\ (9) Tactical Plan;\\ (10) Media Plan;\\ (11) Implementation;\\ (12) Budget;\\ (13) Evaluation and Control\end{tabular} &                   \\ \hline                                
\end{tabular}          
\end{table}

%Method
\section{Method}\label{method}
The research is based on both primary (exercised for this purpose) and secondary (already available) data collection. The primary data is collected by two surveys - one for the current waiting list of users eager to try Hunchat, another for the general public - and a structured interview with the founder and current CEO of the company.  The secondary data is collected from the current communication of Hunchat and its competitors (twitter, instagram and online advertising). All collection tools are indexed on the appendix.


	
%The Plan for Hunchat
\section{The Plan for Hunchat}
\cleardoublepage
\newpage	

%References
\addcontentsline{toc}{section}{\numberline{}References}
\printbibliography
\cleardoublepage

%Appendix
\addcontentsline{toc}{section}{\numberline{}Appendices}
\appendix

%CEO Interview
\section{Founder's Interview} \label{ceo}

The following questions refer to the context surrounding Hunchat - identifying competition, defining strategical standpoints, recognising the market and analysing trends. \par
(1)Who are Hunchat's direct competitors?
(2)What market are you in?
(3)What are the current trends you spot on Social Network's communication?
(4)What need does Hunchat satisfy?
(5)What is the segment you want to reach?
(6)What is Hunchat's positioning?
(7)Who is the target audience?
(8)What is Hunchat's revenue model?

Regarding the brand Hunchat. \par
(9) Describe the product. What are its benefits and functions?
(10) What is different about Hunchat comparing with competition?
(11) What are the brand values?
(12) How is Hunchat distributed? Is it the same as competition?
(13) Define Hunchat with 3 words.

Finding the intended direction for Hunchat's further communication. \par
(14) What do you want to communicate?
(15) What are your goals concerning communications?

%Waiting List Survey
\section{Questions for Hunchat's Waiting List}\label{qwait}

Understand the current relationship between brand and audience.\par
Concerning consumer. \par
() How did you get to know Hunchat?
() What other Social Networks do you consider for the same needs Hunchat satisfies?
() Which of these channels seem more fit to learn about Hunchat?
() What Social Media do you use?
() What do you use Social Media for?


Concerning the brand.\par
() Define Hunchat with 3 words.
() Which of these sentences better describes Hunchat?

The survey ends with collection demographic data about the respondants.\par
() Age
() Sex
() Education

%General Public Survey
\section{Questions for General Public}\label{qtarget}

() What Social Media do you use?
() What do you use Social Media for?
() Do you feel the need for a Social Network for meaningful discussion?


%Testing Brand Elements
%() This is a logo for an app. What do you think the app is for?
%() There's an app called Hunchat. What do you t

%Communication
\section{Competition's Communication}\label{qwait}
\end{document}
