% !BIB TS-program = biber

\documentclass[11pt]{article}
% \usepackage{lipsum}
\usepackage[margin=1in,includefoot]{geometry}
\usepackage[backend=biber, style=apa]{biblatex}
\usepackage{setspace}
\usepackage{multirow}
\usepackage{mathptmx}

\addbibresource{/Users/tiagomarques/Desktop/hunchat/hunchat.bib}
 \onehalfspacing
 \setlength{\parskip}{1em}

\begin {document} 

%Title page
\begin{titlepage}
\title{Integrated Marketing Communication Plan for Hunchat}
\author{Tiago Marques}
\maketitle
\end{titlepage}

\pagenumbering{roman}

%Abstract
\section*{Abstract}
\addcontentsline{toc}{section}{\numberline{}Summary}
This dissertation is an Integrated Marketing Communication (IMC) plan for the newcomer social media app called Hunchat.
  \par
 \textbf{Keywords} IMC, Hunchat, Social Media
\cleardoublepage

\thispagestyle{empty}

%Contents
\tableofcontents
\cleardoublepage 
\listoffigures
\listoftables 
\thispagestyle{empty}
\cleardoublepage 
\setcounter{page}{1}
\pagenumbering{arabic}
 
%Introduction
\section{Introdutction}\label{intro}

\newpage	
%Literature Review
\section{Literature Review}\label{review}
\subsection{A Brief History of Integrated Marketing Communications (IMC)}\label{IMC}
The last two decades of the twentieth century have brought several innovations on the way companies were managed. A highly competitive environment required more creativity on the strategic side of business managers in order to stay relevant. Integrated Marketing Communication (IMC) is one of those innovations whose beginnings go back to the eighties, but whose mainstream popularity only came in the nineties and has ever since influenced over marketing communication decisions. In a world where advertising was the center of all marketing practice, IMC plans were surely a competitive advantage for its users. Its results came to break the ruling paradigm - in 1985 the great majority  of companies' marketing budget (75\%) was spent on advertising. That number decreased to 25\% in 2005 - meaning along those 20 years resources were gradually more widely spread across all communication channels (\cite{holm}).  

Ever since multiple definitions for the term have surfaced. Some simplifying it to the mere management and control of all market communications, others defending it also implies ensuring the brand positioning, personality and message are all congruent, and some even adding its need to be the most efficient (the best use of resources), economical (at minimum cost), and effective (achieving maximum results) strategy (\cite{smith}). The main idea behind it is - as the name suggests - a consistent delivery of a chosen message across all market communications or as  \citeauthor{kitchen} (\citeyear{kitchen}) argued - the goal is to create a "'one-voice' brand phenomenon" (p. 19) that can occur at one or more of the different levels of integration: (1) vertical - the assimilation of marketing goals and corporate goals; (2) horizontal - marketing communications aligned with management functions; (3) marketing mix - having the mix (price, product and place) following promotion decisions; (4) communications mix - tools guiding customers consistently; (5) creative design - executed accordingly to the product positioning; (6) internal and external - assure all departments and employed agencies work based on an equal plan and strategy; and (7) financial - the use of the budget as most effective and efficient as possible (\cite{smith}). 

The beginning of the second millennium reinvigorated IMC with a more strategical approach instead of the widespread tactical one. Instead of focusing on short-term goals, companies began to grasp the perks of the competitive advantage given by thinking long-term. Organisations were now more focused on what was their identity, profile and overall image and made sure the scope of organisation's activities reflected what their owners wanted them to be like (\cite{holm}). Ever since the development of a wide new range of technological advancements - mainly drawn by the widespread use of the internet - gave rise to brand new way of communicating which impacted almost every aspect of human civilization meant the entire marketing communications system was no longer tilted in favour of the seller (\cite{kliatchko}). The fabric of communication was changing and the number of new media growing - computers, smartphones, tablets, search engines, social media, online forums...  Just as three notes can compose some songs but twelve notes can make compose any song, the rise of new media means more and different opportunities to communicate. The emphasis shifted from unilateral mass communication to bilateral targeted communication. What do such drastic changes on the mediatic landscape mean for IMC? Is integrated communication still important in this new world?

 \subsection{IMC in the Digital Age}\label{digitalage}
 
What once was an advantage may have become a disadvantage. The rise of a new diverse collection of IMC options created a lot of new problems on the integration of messages, furthering the misalignment between strategy and tactics (\cite{holm}.) The huge impact of the digital revolution on culture and society has manifested mainly by the way we communicate. The creation of social media completely changed the ruling paradigms of mass communication. No one could ever predict the groundbreaking network based media that is now so ubiquitous (or at least at this extent), but awareness of the power of technological advancements were quite common on researching IMC. Some expected the development of a media anarchy (\cite{solomon}), others that it would need to account user generated content (\cite{ananda}). Nevertheless the core model regarding the more tactical and strategic issues (such as \citeauthor{schultz} in \citeyear{schultz} or \citeauthor{duncan} in \citeyear{duncan}) although designed with traditional media in mind still applies to modern communication channels such as social media for their conceptual frameworks do not raise any significant barriers in its implementation. That is not the case for much of the traditional IMC models being developed when most media interactions were outsourced since they do not consider the implications of message delivery in so many, volatile, and widely specific channels. As \citeauthor{mcluhan} famously pointed out "the medium is the message" (\citeyear{mcluhan}), and with so many different media each with its unique specificity (history, context, surrounding communities, mediatic traits...) it is quite challenging (if not impossible) to encompass the same core message across all of them. The problem is no longer only which message to tell, but also which medium suits it best. It isn't only this larger amount of touch points that's threatening, but also their nature. Direct messaging and user generated content like comments, publications, stories... shift or at least balance the control over the message from the organisation towards the consumers, only aggravating the already difficult task of communicating consistently. Social Media also introduced a blurring of marketing functions into one - sales, promotion and services are now mushed together and sometimes indistinguishable from each other (\cite{valos}).  

While raising all of these implementation problems for IMC, social media has also brought an infinite amount of new and different opportunities to communicate. The real time feedback from consumers, the ability to better monitor the performance of campaigns and its impacts, the larger amount of analytics and insights collected from consumers can all strongly contribute to a better and more informed IMC decision making. It also forms a much more efficient, effective and economical way of communicating since it's delivered only at the targeted people - highly reducing the churn rate - for the desired amount of time, at the best context with as much specificity wanted for a small parcel of the cost of traditional media. The engagement from consumers provides an ongoing interaction, giving more agility and possibilities of integration. If there's a party winning for its presence on social media it is the advertisers. The abandonment of a sporadic one way communication should be seen not as a threat to IMC, but as an improvement on its ability to form a continuous, transparent and free dialogue between customers and organisations (\cite{ananda}). 
 
It is also important to point out that modern communication on social media is usually mediated by intelligent black box algorithms by which many of market decisions are made - whether it is by high frequency stock trading (\cite{hft}) or personal feed curation (\cite{algo}). What this means for IMC is that it should account for these new artificial 'readers' of content as to achieve the maximum exposure, considering its possible effects on all stakeholders without jeopardising the original message.  

 \subsection{The Importance of Communication for Social Networks}\label{SN}

A Social Network is not only a medium of communication but it also needs to be communicated itself. The market of social media apps is quite large - there are fifty thousand new competitors every month (on the app store) all trying to be the next facebook (SOURCE) - and so standing out from the huge crowd is a challenge and communication is a crucial help.
 
Analysing the narratives that big players in the industry have used to rise to the top there is a clear commonality: counter-culture. That is not unique for social networks but for most tech companies whose beginnings were viewed as the ultimate hope for social change. The fact that digital technology seemed free of human fragilities, it promised to solve most human made societal problems. Even though the hopeful prospects for tech might have not fully realised, all major social networks still have each broken some ruling paradigm - instagram revolutionised the way pictures are shared online, facebook changed the way we share our personal information on the web, whatsapp took online chatting and made it personal - and most made sure to communicate that revolution (\cite{evil}).
 
What really makes communicating a social network so different from any other business is the importance of reaching the right people. The product sold is not only the tool itself but mainly the people who use it (hence the name 'social network') - one does choose which social media app to use mostly based on its participants. That not only implies a necessity of a minimum amount of people using the product, but also a concern for which people exactly. A common strategy to solve the participation problem is to ensure exclusivity. Facebook has achieved it by exclusively accepting new users who were directly invited by those already using it (imitating universities' social clubs) (\cite{zucked}). This strategy not only ensures that new users already have other users to connect with (which is the main goal of a social network), but can also serve as a form of buzz marketing - one must always wonder what's happening behind closed gates. The success of clubhouse has just proven how effective opting for this path can still be.
 
After reaching a solid base of users the next big challenge in communicating a social network is to stay relevant.  There are plenty of social networks who manage the difficult task of reaching a large amount of users but then lack in means to keep them interested, in fact 80\% of all apps get deleted within the next three months.

For the ones who do overcome all these hurdles it then arises the bigger struggle. As it has been shown by almost every social network achieving this long-term widespread popularity, the big tech monopolies always want a piece of the pie and will try to acquire it. That was the case for WhatsApp, Instagram (bought by Facebook), Youtube (bought by Google), LinkedIn, Discord (bought by Microsoft), Tumblr (bought by Yahoo) or Twitch (bought by Amazon). The ones who are brave enough to resist and persist on continuing on their own usually end up being copied and then have the difficult task of competing with their own product with the added hassle of the competitor's ownership of a larger, more engaged, and already fully developed network of people. That was the case for TikTok, Snapchat (copied by Facebook) . Then there are the small few cases who despite this arduous environment still manage to retain its network. That is the case of Twitter and Reddit.

All this raises three questions: (1) How can a social network differentiate from the big ones as a newcomer?; (2) How can a social network stay relevant after reaching a large audience?; (3) How can a social network stand out even when larger competitors are able to use the tools that differentiated them in the first place?; All questions in which communication can be a crucial part of the solution. Therefore such hazardous challenges require the use an IMC conceptual framework capable of addressing all of them.

 \subsection{IMC Frameworks}\label{fw}
 
The ever changing nature of technology whose velocity has been specially fast in the last two two decades created a morphing landscape more fit to those who adapted their strategies accordingly. 	That adaptation can be seen through the evolution of IMC. The almost forty long years of its existence made it gradually evolve to ever more complete frameworks. We have come a long way since the view of \citeauthor{nowak} (\citeyear{nowak}) passing through the \citeauthor{kliatchko}'s people based view framework (\citeyear{kliatchko}) and \citeauthor{valos}  (\citeyear{valos}) already devised a decision making framework that integrates social media in the IMC process. 
 
  \subsection{Conclusions}\label{conclusions}
  
  Integrated Marketing Communications started at the end of the twentieth century with the goal of integrating messages across all different media. Ever since it has morphed according to the mediatic changes brought by technological progress of the communication landscape. Social media and the current way of communication poses a threat for IMC - integrating a message across so many different and specific media with the aggravation of having way less control over what’s being communicated is extremely difficult - but also brings new opportunities to communicate - real time feedback, better monitoring capabilities, more consumer insights - more efficiently, effectively and economically. The Social Network market is extremely competitive which makes it hard to stand out. Counter-culture is a common narrative for newcomer social networks. The users make the product and exclusivity is a way to solve the participation problem. Staying relevant is the next challenge. Being bought or copied by the big tech monopolies follows. The goal goes: differentiate yourself, stay relevant, compete with your own product. IMC frameworks evolved with the landscape - some are purely conceptual, some integrate specific (like social media). 
 


%Conceptual Framework
\section{Conceptual Framework}

The superiority \citeauthor{clow}'s framework for IMC is clear - not only for its complete and robust body of contextual analysis and practical application tools, but also for its strategic versatility (and more abstract nature) enabling an easy integration of social media. Even though this framework excels in assuring an overall communication integration, it lacks in considering how the use of brand elements in accordance with the key message might be crucial to a full integrated marketing communication (and specifically on this context as seen on \ref{SN}). For that reason I will encompass \citeauthor{kliatchko}'s brand and consumer auditing phases from the framework he proposed in \citeyear{kliatchko}. The conceptual framework for the IMC plan for Hunchat will then be as represented in table \ref{table:concept}. 

\begin{table}  [htb]
\centering
\caption{Conceptual Framework}
\label{table:concept}
\begin{tabular}{ll}
Author             &  Plan Phase(s)    \\ 
\hline
\cite{kliatchko} & \begin{tabular}[c]{@{}l@{}}(1) Consumer Auditing;\\ (2) Brand Auditing\end{tabular}                                                                                                                                                                                         \\ \hline
 \cite{clow}  & \begin{tabular}[c]{@{}l@{}}(3) Internal Analysis; \\ (4) External Analysis; \\ (5) SWOT Analysis; \\ (6) Goals; \\ (7) Strategy; \\ (8) Messages; \\ (9) Tactical Plan;\\ (10) Media Plan;\\ (11) Implementation;\\ (12) Budget;\\ (13) Evaluation and Control\end{tabular} \\ \hline                                
\end{tabular}          
\end{table}

%Method
\section{Method}\label{method}
The research is based on both primary (exercised for this purpose) and secondary (already available) data collection. The primary data is collected by: (1) a structured interview with each founder of the company (\ref{ze}, and \ref{ernesto}); (2) a focus group with some members of the current waiting list of users eager to try Hunchat (\ref{fwl}); and (3) a survey for the general public (\ref{qgp}). The secondary data is collected from the current communication of Hunchat and its competitors (twitter, instagram and online advertising) (\ref{comp}). All collected data is indexed on the appendix (\ref{data}).

\subsection{Interview with Founders}
An interview was done with each of the two founders seeking to answer questions about all the different phases of the plan. It is part of the qualitative research done in order to serve as guide for the final IMC plan. The questions were constructed as to respond to all the needs for information for each phase as the \ref{table:founder} shows. The interviews itself were semistructured, meaning the script was open for changes during its course (adding or removing question, switching their order...) which allows it to morph to the desired direction. Both interviews were done via video call and had approximately half an hour in duration. The full transcripts are available on the appendix - on \ref{ze} for José and on \ref{ernesto} for Ernesto. 

\begin{table}[htb]
\caption{Questions for Founders}
\label{table:founder}
\centering
\begin{tabular}{lll}
Author                      & Plan Phase             & Question                                                                                                                                                                                                                                                                                                                                                                                                                         \\ \hline
  &  Context & Why did you create Hunchat? Can you tell its story? \\ 
\cite{kliatchko}             & (2) Brand Auditing     & \begin{tabular}[c]{@{}l@{}}(1) Describe the product.What are its benefits and functions?\\ (2)What need does Hunchat satisfy?\\ (3) What is different about Hunchat comparing with competition? \\ (4) What are the brand values? \\ (5) How is Hunchat distributed? Is it the same as competition?\\ (6) Define Hunchat with 3 words.\\ (7) What does Hunchat mean to you? \\ (8) Why are you helping building it?\end{tabular} \\ \hline
\multirow{8}{*}{\cite{clow}} & (3) Internal Analysis; & \begin{tabular}[c]{@{}l@{}}(9) How are you currently communicating and why?\\ (10) What is the revenue model for Hunchat? \\ (11) How and how much is it priced?\end{tabular}                                                                                                                                                                                                                                                    \\
                            & (4) External Analysis; & \begin{tabular}[c]{@{}l@{}}(12)Who are Hunchat’s direct competitors?\\ (13)What market are you in?\\ (14) Identify the pressure groups for Hunchat.\\ (15) What do incubation programs mean for the future of Hunchat?\end{tabular}                                                                                                                                                                                              \\
                            & (5) SWOT Analysis;     & \begin{tabular}[c]{@{}l@{}}(16) What are Hunchat's streghths? \\ (17) What are its weakness? \\ (18) Where do you see oportunities?\\ (19) What might threaten Hunchat's success?\end{tabular}                                                                                                                                                                                                                                   \\
                            & (6) Goals;             & (20) What are your goals concerning communications?                                                                                                                                                                                                                                                                                                                                                                              \\
                            & (7) Strategy;          & \begin{tabular}[c]{@{}l@{}}(21)What is the segment you want to reach? \\ (22)What is Hunchat’s positioning?\\ (23)Who is the target audience?\end{tabular}                                                                                                                                                                                                                                                                       \\
                            & (8) Messages;          & (24) What do you want to communicate about Hunchat?                                                                                                                                                                                                                                                                                                                                                                              \\
                            & (10) Media Plan        & (25) Where is it mandatory for the brand Hunchat to be?                                                                                                                                                                                                                                                                                                                                                                          \\
                            & (11) Budget            & (26) What is Hunchat's communications budget?                                                                                                                                                                                                                                                                                                                                                                                    \\ \hline
\end{tabular}
\end{table}

Although there's a clear vision and mission behind Hunchat, the path to get there is still fully open. It became clear that the ultimate communication is goal is to explain what is Hunchat. Zé seems more inclined towards a global view when he talks about his desire for new people to meet and start great things over Hunchat - “A story I like to tell myself is that someone will meet someone on Hunchat and start a business, just as we did on twitter. I want that other people are able to do it in a platform that we built.” Ernesto is more inclined towards a more local view mentioning how he wishes for people to connect with their friends and family over the app. Both agree that their mission is to “be the best place to have conversations online”.
\pagebreak
\subsection{Focus Group with the Waiting List}
The second part of the qualitative research is a Focus Group with eight signers of the waiting list (that is already passed the two hundred) of people eager to use Hunchat. The fact that these people are already familiar the brand and like enough to subscribe to their newsletters and offer their time to test the app as soon as it out make them ideal to (1) audit the brand and the consumer. The question were therefore constructed mainly for that purpose. There also can add some research on their (2) social media use habits, and understand (3) which messages they seem more prone to and (4) in which channels should they be communicated.

\begin{table}[htb]
\caption{Questions for the Waiting List Focus Group}
\label{table:focus}
\centering
\begin{tabular}{lll}
Author                             & Plan Phase             & Question                                                                                                                                                                                                                                                                                                                                                      \\ \hline
\multirow{2}{*}{\cite{kliatchko}} & (1) Consumer Auditing  & \begin{tabular}[c]{@{}l@{}}(1) How did you get to know Hunchat? \\ (2) Why is Hunchat different?\\ (3) What social media do you use?\\ (4) What do you use Social Media for? \\ (5) Do you discuss your ideias online? \\ (6) Do you communicate using video regularly? \\ (7) Do you feel the need for better communication apps? \end{tabular}                                                                                                                  \\
                                   & (2) Brand Auditing     & \begin{tabular}[c]{@{}l@{}}(8) Describe the product. What ate its benefits and functions?\\ (9) What need does Hunchat satisfy?\\ (10) Define Hunchat with three words.\\ (11) What does Hunchat mean to you?\\ (12) Why did you join the waiting list?\\ (13) What is different about Hunchat comparing with competition?\\ (14) What are Hunchat's brand values?\end{tabular} \\ \hline
\multirow{2}{*}{\cite{clow}}      & (4) External Analysis; & \begin{tabular}[c]{@{}l@{}} (15) Do you have meaningful discussions online? \\ (16) Where or why not? \\ (17) Which social media do you use? \\ (18) How do you use it? What's good and what is missing? \end{tabular}                                                                                                                                                                                                                                                                                                                \\
	& (8) Messages & \begin{tabular}[c]{@{}l@{}} (19) What do you think Hunchat should communicate? \\ (20) Are you concerned about privacy issues? \\ (21) Do you feel the content you usually consume is meaningless? \\ (22) Do you feel you cannot fully express your ideias on \\ existing social media apps? \\ (23) Do you feel synchronous video calls are inconvenient? \end{tabular} \\
                                   & (10) Media Plan        & (24) Which channels seem more fit to learn about Hunchat?                                                                                                                                                                                                                                                                                                       \\ \hline
\end{tabular}
\end{table}

\subsection{Survey for the General Public}
A survey to the general public can tell us (1) if there's really a need for an asynchronous video social network; (2) what people look for in a new social media; (3) from which channels can we reach them; (4) which potential narratives resonate better.


%The Plan for Hunchat
%\section{The Plan for Hunchat}

%\subsection{The Problems of Current Social Networks}\label{problems}
%- Impact on well-being - addiction, 
%- Privacy and Data Collection
%- Gatekeepers vs Free Speech (centralized power over communication)
%- Monetization.
%- Anti Trust effects: big tech deciding what gets to the users
%- Echo chamber from algorithm curation of info
%solutions
%- New alternatives- Nebula for youtube, direct payemnet to creators (patreon). Hunchat.

\cleardoublepage
\newpage	

%References
\addcontentsline{toc}{section}{\numberline{}References}
\printbibliography
\cleardoublepage

%Appendix
\addcontentsline{toc}{section}{\numberline{}Appendices}
\appendix

%Collected Data
\section{Collected Data}\label{data}
\subsection {Interview with José} \label{ze}
The first interview was done in... with José Gomes one of the founders and the current designer of Hunchat. With his consent.

(1) Why did you create Hunchat? Can you tell its story? \textit{Hunchat started when Twitter launched a new feature called ´fleets' that is basically instastories or snapstories (whatever you wanna call them) for Twitter. I looked at it as wasted opportunity for Twitter to use video in a worthwhile way, and so I ranted about it on Twiiter and Ernesto saw it. He came to me saying we should do something about it. One day we picked up a Figma file} - referring to the app designing computer software -\textit{and started an initial draft of what later became Hunchat. Our goal is to enhance online conversations in the most human way possible. Our mission is to be the best way to have conversations online. Connect people, connect projects in the same way I did with Ernesto over Twitter. Create a good place por people to talk.} 

(2) Describe the product.What are its benefits and functions? \textit{Our product, Hunchat, is a video based social network focused on the exchange of ideias and conversations between users. As I already said, the goal is to be the best place for online conversation and we are a video platform that uses the thread system (just as Twitter) - when someone posts a video, then waits for responses from other people. All content is video, and now in the beginning, it is open and public - just as the timeline in Twitter, where people can interact with each other and create parallel conversations - but in the future there will be private chatrooms too.} Why private chatrooms? \textit{We have tested in little WhatsApp groups and such, and there is always a better dynamic when people already know each other and speak only for one another instead of speaking to te world in general. It is different sharing something with a friend than sharing on the internet with everyone. So we're first testing the concept publicly and then create that individual groups dynamic or even one-on-one.}

(3)What need does Hunchat satisfy? \textit{It satisfies the need for online communication, of interacting with people in more human way.}

(4) Knowing that there are already so many different apps trying to satisfy the same need. What is different about Hunchat comparing with competition?  \textit{Right now, there are only tow ways of communicating online: asynchronously - text messages, tweets, audio messages, video messages are appearing more but there's no platform designed specifically for it; Or synchronously - such as the call we are doing right now in which the conversation is more organic, we can see each other, it is a different interaction. We want to create a similar live interaction without the need of synchronism. No need to schedule. Happening in such an organic way as we do messages but as good as an interaction as live. } 

(5) What are the brand values? \textit{Our values are the connection between people. Transparency - our code is open source, anyone can see our code, we are committed to collect as minimal information possible, and have no plan of selling information as opposed to any other social network. We value privacy, openness, and connection between people.}

(6) How is Hunchat distributed? Is it the same as competition? \textit{Yes, the direct access will be in the app store in the next two months. First only in iOS, and only later will we think about android as a way managing resources.}

(7) Define Hunchat with 3 words. \textit{Transparent is important. Intimate. And the ideia of getting two people to start something new is also important. I don't knwow how to describe it.} Serendipity, maybe? \textit{Something like that.}

(8) What does Hunchat mean to you? \textit{For me, even though I'm repeating myself, is the online conversation between people. I meet a lot of people online - mostly on twitter, it was the case for Ernesto - and I believe it is really useful to create new platform where people can do it, that I also can do it in a different way. Simply, for there to be a better option for communication with other people online. For me it is also specially important, being my first company. Creating a project of this dimension for a personal project is of course important.}

(9) How are you currently communicating and why? \textit{Right now, we are only two people making it difficult to make a huge communication. So we are basing it on 'building in public'. Right now, our pipeline is: whatever is happening, we write a blog post about it, and then recycle the blog post in other content across other channels. So, from the blog posts comes all of the rest: Instagram posts, LinkedIn posts, tweets...} Is it all done online, then? \textit{It is all online communication, yes.}

(10) What is the revenue model for Hunchat? \textit{Being a B2C with a social network dimension, in the early life of this type of startup it is still too soon to think about monetisation. But right now what we are planning is making it freemium. A goof example of that model is Tinder. Usually people think about Tinder as a free app, but there is a big base of paying users that support the whole app, that are paying for extra tools. } What tools will they be, then? \textit{What we're going to do to make tools that people actually want is to make them free first, and then wait for the feature requests - "we want this", "we want that" -  and then we build them accordingly and the put them behind a paid plan. We are never going to take away things we already gave for free. There is already a big history of companies making it, and receive an enormous backlash from their community. So, as new features are requested we are going to get them behind a paywall.} And what kind of features will they be? Introducing paid features won't somehow disrupt  the democratic values you're trying to associate Hunchat with?\textit{They will be features that give the paid users no advantage over other users. For example being able to double speed the videos.}

(11) How and how much is it priced? \textit{They aren't defined yet, and there's nothing to gain on speculating about it now.}

(12) Who are Hunchat’s direct competitors? \textit{Direct competitors in terms of having the same product, we still have not identified any. Indirect competition are applications like '100mentors' - a small app that makes something similar to what we do for the mentoring area. Someone publishes a question and then a mentor shoots a video responding to it. And there are other niche apps appearing that also use asynchronous video. In a more general sense a big competitor would be Twitter. }

(13) What market are you in? \textit{In the market of social networks.}

(14) Identify the pressure groups for Hunchat. \textit{Simply I can't identify any.} 

(15) What do incubation programs mean for the future of Hunchat? \textit{We are in two different incubation programs. Startup Lisboa in Portugal and Hook in France, Paris. Both are remote, not only due to covid.} - referring to the global pandemic experienced in the time the interview was done that shifted most of social activities to be done remotely - \textit{Hook made a direct investment of fifty thousand euros for five por cent of the company. And keep up with the company for 5 months. We have workshops, they support us in whatever we need. And we have demo days where we present to investors in order to have a new insights. It is like a Y combinator but in Europe.} - mentioning the world's largest incubator of startups situated in the US -\textit{ It is the same model. Startup Lisboa help us in a different way. They don't invest directly in the company. They only give us access to important groups of the Portuguese and European markets, and mentoring sessions.} Are there any side \textit{None. There is a mutual interest, our only obligations is to send them investment reports later, and balances for five years. We. have total freedom.}

(16) What are Hunchat’s strength? \textit{Being a small company - being a small team our costs are few. Focus. Innovation in the asynchronous video area.}

(17) What are its weakness? \textit{Being small is also a weakness in many ways. Lack of resources slows down the process, we are not at the speed we wanted to.}

(18) Where do you see opportunities? \textit{Our opportunities are in the social space, and being the first asynchronous video for the masses.}

(19) What might threaten Hunchat’s success? \textit{Competition will surely later be one, but it won't kill us now. Not being able to be fast enough, entering too late on the market to catch people's attention and have a satisfying retention rate.} What do you mean by fast enough? \textit{In the startup world, everything is very fast. For you to have an idea, for the hook program it is mandatory (not mandatory more like a moral obligation) to publish something new every week. Every week there needs to be a new change in the product, a new advancement, something new based on people's feedback. And then track what went wrong, what we learned. It is really important that with the money we have, we have a limited amount of time without new capital and without making money, so when the time comes we need to be the best we can. For that we need to keep iterating. Specially in apps like ours usually the retention rate in the begging is awful (or if you're lucky it is good), but the goal is to keep getting bigger a percentual point at a time. Even twitter looses 70\% of its users. Periscope, Vine... all died because of this.} It then diverted to a discussion about the rise of social networks that is not really relevant for the plan.

(20) What are your goals concerning communications? \textit{Our goal right now is to communicate this... How do I put it? It is to communicate what the app is exactly. Being a product for the final consumer, and being the social network such an abstract concept it isn't easy for people to understand right away what it is, without experimenting it first hand. So our goal now is to find a basic and simple form of explaining exactly what is Hunchat.}

(21) What is the segment you want to reach? \textit{There are two different responses to this question: one for now, and one for later. A social network is as good as the content that users generate there. So there comes the problem of the chicken and the egg. If there isn't content, there won't be users. If there are no users, then there isn't content. So our goal is to be good to use it even with only a hundred people. And we're going to do it by entering in already existent online communities and convince them to use Hunchat. Enabling a closed system with an organic growth inside each community until they are satisfied and us convinced they won't leave. Only then explore a more open market.} Where are those communities? \textit{We are going to test soon with one. It is called inter-intellect. It is basically a corner of the internet full of nerds like me which like weird things and make saloons in which someone presents a theme and gives a mini-workshop that ends up in a three hour long conversation. Later, we can present ourselves more like a social network -in a more direct approach - just like in the street you can approach any person,}

I skipped the question '(22) What is Hunchat’s positioning?' since it seemed to be already answered. (23)Who is the target audience?\textit{For now, it is people that are already used to making video calls, that are already used to talk with other people online - and so our focus is on online communities,}

(25) What do you want to communicate about Hunchat? \textit{The importance of proximity when talking with people online.}

(26) Where is it mandatory for the brand Hunchat to be? \textit{Twitter, Linkedin, Instagram and blog posts.}

(27) What is Hunchat’s communications budget? \textit{Right now, we have no budget allocated for communication.}

(28) Who are the stakeholders? \textit{We are B2C directly and we have no advertising, our main stakeholders are the clients, ourselves, and our investors.}  Which will be most affected by Hunchat? \textit{People who use Hunchat and meet new people on it. A story I like to tell myself is that someone will meet someone on Hunchat and start a business, just as we did on twitter. I want that other people are able to do it in a platform that we built. } Who controls the resources? \textit{Me and Ernesto.} What motivates each stakeholder? \textit{For users it is clear. For me it is the fact that it is my own business and obviously want it to be a success. For the investors it is obviously a financial interest.}

\subsection{Interview with Ernesto} \label{ernesto}
The following interview with Ernesto Gonzales is one of the founders and the current main developer at Hunchat. With his consent.

(1) Why did you create Hunchat? Can you tell its story? \textit{The creation of Hunchat comes from the frustration that I and Ze had on social media. We feel that text usually doesn't capture the essence of what we're trying to communicate. We think that video is the best format to communicate, but it comes with the problem of synchronisation of calendars. For example, I am cuban, and my grandparents live in Cuba. They call me a lot of times over WhatsApp and I can't pick up because I am working. And the opposite is also true, I call and they can't pick up. Video messages solve that problem. I think that when we have all the components of verbal communication - body language, facial expressions, intonation - video is so much more closer from a face-to-face interaction than any other format. The ideia surged when twitter launched 'fleets' - which is basically stories for twitter - and Zé published a fleet questioning what uses fleets could have and one of them was this type of communication we're looking for on Hunchat. People talking about stuff just like on twitter, and with their friends on WhatsApp but with video messages. Just like a vlog - on one side for being public, and on the other for being a video message. From there we started building on the ideia forward.}

(2) Describe the product. What are its benefits and functions? \textit{Hunchat is a social network that allows its users to communicate with their friends and family using video messages. Its main benefit is the 'asynchronisation' of calendar, that problem is solved from the root. We no longer need to be both available at the same time, it allows the video format and can even helps manage the tiredness of video calls - nowadays there are video calls for everything. And then there will be the benefits from the content the users will get on Hunchat. But exactly what only time will tell, it is to soon to talk about it.}

(3) What need does Hunchat satisfy? \textit{The necessity of communication. Sometimes it's difficult with consumer products - they don't solve a problem, they take advantage of opportunities not yet explored. The problem only surges when we give and then take. Now we're giving, only then will we see what the problem is. It always comes back to the same ideia - creating a better communication context for family, friends and talking openly with the general public. I can't speak about a specific problem, we're working on improving the opportunity that video is, that we think has been under-explored.}

(4) What is different about Hunchat comparing with competition? \textit{The more direct competition might be TikTok. People always associate due to the video format. If we think about it through architectural standpoint they are quite similar. The content on TikTok is limited in time - one minute only per video - that has huge impact on the type of content shared there. On Hunchat there's no limit, just as on a video-call. We could even forget the fact that TikTok has the music, and dances. The time limit alone makes it absolutely different. The biggest difference is the content. They are building a global community, we are strengthening the local community. Our focus is friends and family. You upload a video - you go somewhere, you're showing where you were at,  you are sharing a moment with your friend, announcing some news, sharing your thoughts for the future - and show the same type of thing you would on a video-call with your friends or family. That's it.} But how does exactly an asynchronous video conversation work? Is it like voice messaging? \textit{It's exactly that. Just like a voice message, but in video.}

(5) What are the brand values? \textit{We are looking for meaningful communication. We are not looking for what already out there on instagram - influencer media, content for it's own sake just for followers... We are looking for the same type of core audience as on early Instagram. You sharing with your ideas with your friends.}

(6) How is Hunchat distributed? Is it the same as competition? \textit{When we talk about apps, the goal is usually the App Store for iOS, and the Google Play for Android. Right now, it still isn't available on the app store, but in the next few weeks it will be there. The app is already ready to launch, i just finished it yesterday. We were supposed to have launched it some weeks ago, but we had some minor problems that are now solved. We are now submitting it to Test Flight,} referring to apple's platform for testing beta versions of newcomer apps. {...and then App Store.}

(7) Define Hunchat with 3 words.  \textit{Friends, meaningful - the thing we're always after, and sharing.}

(8) What does Hunchat mean to you? \textit{It always comes back for me wanting to talk to my grandparents. It means an opportunity to obtain a better communication dynamic between me and my family. I have brothers in Angola, grandparents, cousins and uncles in Cuba. An app like Hunchat will make communication in cases like mine easier. }

(9) Why are YOU helping building it? \textit{Because I feel inspired by the project, and what it can become. I'm curious in what it will transform to. It is a challenge, and we like challenges.}

(10) How are you currently communicating and why? \textit{The communication might be the thing we've been having more difficulty on. Zé is a designer, I have a more technical background, so this is not our area of expertise. What we've been trying to do is: in the beginning we've gone from a communication inclined towards 'what is Hunchat'. Something like 'video based social app for asynchronous conversations' but that is highly descriptive. What we noticed was that 'asynchronous' is a word most people don't even know what it means and that it is too much of a 'what?'. Now we're trying to go from 'what' to 'why'. We're now trying to change our communication to the reasons why we are creating Hunchat, and transmit our vision. But I think we're not there yet. We've been saying that Hunchat is an app for communication with friends and family, for meaningful conversations - that type of language. It always ends up in meaningful conversations.}

(11) What is the revenue model for Hunchat? \textit{We had one already developed, but a problem raised. In this type of products, you need scale, so it isn't worth spending time thinking about that now. But our ideia is a subscription model of circular economy.  A freemium app in which paid subscribers enter in the circular economy. For example, 20\% of my time in the app I am consuming your content and the other 80\% I am consuming Mary's content. From the ten euros I paid for the subscription, two will go to you and eight will go to Mary. 20\% of those transactions go to Hunchat. While the free users can still use the app for free. We don't like data. We are trying to avoid collecting it, because we think that in general data will worsen the user experience.}

(12) Who are Hunchat’s direct competitors? \textit{People usually mention TikTok. And if we talk about ease of integration in the app, TikTok is in fact who has the easiest integration. But we have completely different visions: theirs is entertainment, ours is communication. So a better answer would be facebook for their fame of always copying new players, and resources (like engineers), and so on. There may be other competitors we haven't heard about yet, and those worry me way more.}

(13) What market are you in? \textit{Our strategy is now based on entering in different communities, one at a time. We are now entering in a community called inter-intellect - a community for sharing ideias and knowledge - that is global.  So we cannot talk about local markets, like the American market, or the European market. Let's just say the market of online users.}

(14) Identify the pressure groups for Hunchat. How do they affect you positively and negatively? \textit{It is too soon to talk about those. We started testing the app two months ago, but later we started reconstructing it in another programming language to make the performance better. We only used ten more people, all friends, it isn't even that used yet. It is too soon to discuss all that. }

(15) What do incubation programs mean for the future of Hunchat? \textit{The first one made the project real. It went from a two computer kid's project to two kids creating a startup. It is somewhat different. It helps with morale - when someone asks you 'what are you doing?' you no longer have to say 'I'm working on a project', you can say 'I am working on a startup'. What I mean with this is that you give another meaning, another weight to Hunchat. By entering in the entrepreneurs' community we met a lot of interesting people that can teach a lot of things. We are just starting and there's so much for us to learn, and that's important. The french market is very large on consumer products, their startups ecosystem is second the largest in Europe (after London if I'm not mistaken). It is huge and that's an advantage for us, being in Portugal being able to enter there. We're automatically introduced to investors who have more experience with these types of product. That's extremely important. Meeting people that have a good understanding of our market that can give us advice. Without even mentioning the money that they invested on us. Even without that part, it would still be an amazing opportunity.}

(16) What are Hunchat’s strengths? \textit{Our vision - making Hunchat the best place for people to communicate online with friends and family.}

(17) What are its weaknesses? \textit{We have a lack of resources. We do not have a lot of experience building this type of products. I'm the only programmer and run with big struggles. Building an app alone is already difficult, building a video app is one of the most difficult thing you could do.} He then gave some technical examples of hardship that despite being quite interesting do not regard this communication plan. {Our biggest weakness must be the lack of technical expertise. But with our motivation, at the end of the day we will have that experience.}

(18) Where do you see opportunities? \textit{In the under-explored space of video communication. For us to communicate with video and take full advantage of its potencial, there's no need to be synchronous. We already went from messages on a dove's paw with months waiting, to the mailman that gives them by hand with weeks waiting, until we got e-mail, and text messages, audio messages, phone calls. We think there's still so much to explore. Hunchat will be there, alongside so many other companies. We believe there's a lot of potencial with video.}

(19) What might threaten Hunchat’s success? \textit{Its success can only be threaten by its own team. In the startup world the biggest enemy is your own team. Right now, it is only me and Zé. It is important for us to accumulate small victories, in order to maintain our morale high. Having ambitious goals, but small and with short periods. Staying focused on our vision. It is extremely difficult to maintain mental sanity while working so young on a startup. We are out biggest challenge, not our competition, not the fact that we are not on Silicon Valley... us and our capability of staying focused are our biggest adversities.}

(20) What are your goals concerning communications? \textit{For now it is finding a language that can transmit our vision. Our vision is there, our mission is there. The struggle is communicating what Hunchat is trying to do, also what Hunchat is. It's like a spectrum. On one side we have the 'what', and on the the other the 'why'. We are trying to move towards the latter, but we haven't found the sweet spot yet. Something we have also noticed is that if we incline too much towards the 'why', people might understand it but then go 'you'll do all this for me, my friends, and family, but how?'. The ideia is: before people actually make the action we want them to do, they should already have an ideia of what Hunchat is. Maybe that can be done just with a screenshot of the app, maybe a video, or photo of someone using it. Only experimentation will tell. But we are on the right path. On the last few weeks we have been walking in the right direction, we've shifting our attention towards textual communication on Twitter and LinkedIn. With time we'll get there.}

(21) What is the segment you want to reach? \textit{In the long run, to be the best place for online communication we need to reach the mainstream, we need to be global. But right now we're reaching a community at a time. Our goal now is Interintellect, we're focused on the niche of their users.}

(22) What is Hunchat’s positioning? \textit{Transparency, focus on content that the users find important and meaningful. It is not for videos of the Rock dancing or for memes - that's entertainment. The important content for us is what originated social media in the first place on the millennium's beginning and that is what our friends and family have to share with us.}

(23) Who is the target audience? \textit{Right now it is a member of Interintellect.} But besides that who is the type of person you want to reach? \textit{I don't believe there's a specific type of people. For example who is the target of WhatsApp? There's none. If we're trying to create a product that allows me to talk with my grandmother then it can be technologically exigent, it needs to be easy to use. Record, upload, watch. It needs to be simple. Not targeted just for young people or old people. It is for everyone. But I'm talking for the long run, maybe within a year and a half. Now we're focused in idea sharing communities of the tech-twitter type. Now specifically Interintellect - a community of approximately ten thousand people. For the next two months they are our focus, and the only users of Hunchat. That's our market. We believe that's the way to go now, because if we try to build to everyone right away, we'll end up building for no one.} And after that? \textit{We are targeting other people who like us are building in public. From there move towards the startups ecosystem - entrepreneurs like us. A similar model to what clubhouse has been doing within the Silicon Valley elite. We are focused on 'build in public' communities - first Interintellect, then maybe startup Lisboa, then maybe a subreddit.}

(24) What do you want to communicate about Hunchat? \textit{I want to communicate that Hunchat is a space for sharing.}

(25) Where is it mandatory for the brand Hunchat to be? \textit{Interintellect. Zé has been living there, meeting upon meeting, to understand what they do, how they think, what they want. We need to know well our target. And because they are our target right now, that's what we've been doing. Hunchat has a mission but there isn't an essence yet. We are in the liquid estate. We're still evolving. It will be our users that will tell us what Hunchat is. The vision is here, the mission is here, but it's them that gonna tell us how to get there.} And after Interintellect? \textit{We don't like data in the app, so I don't believe we should advertise on other social media. It is a question of sharing our vision in other social networks - Instagram, Twitter, LinkedIn, and even word of mouth. We want the experience to be so good that people will invite their friends and family to join Hunchat. It will be always online. I cannot imagine a billboard or a physical event.}

(26) What is Hunchat’s communications budget? \textit{I can't say exactly. Two months ago we made the calculations. It is extremely difficult without having a certain amount of users to know how much you can spend acquiring a new costumer, because you can't know what's the costumer lifetime value, or the costumer acquisition cost. But I can say that our goal for the next six months is to have two people oriented for that. So for a one year's budget that would be one hundred and sixty thousand euros, but only spent on human resources - a marketeer and a designer for example - not on ads or anything.}

(27) Who are the stakeholders? Which will be most affected by Hunchat? Who controls the resources? What motivates them? \textit{Our biggest stakeholder will clearly be our users. But again it is still too soon to specify. We are with our heads outside the hole trying to listen, collect information. It depends on our luck, the path we take. We know our vision, we do not know the path to make it true yet. Even the users don't know - we don't know and they don't know. We need to discover. We also have our investors from Hook. And me and Ze. But it always end up on the users, they will tell us where to go, what they want and we will listen. Hunchat can work with feeds, chatrooms, can be more oriented for public, or more oriented to private. Think about the dichotomy of global and local. Right now I'm telling you it's more local - family and friends - but maybe users will push it more towards the global. We do not know. Our goal is to be the best place for online communication, how depends on the users. They have most of the decision-making power.}

\subsection{Focus Group with Hunchat's Waiting List Members} \label{fwl}

%The survey ends with collection demographic data about the respondants.\par() Age() Sex() Education

\subsection{Survey for General Public}\label{qgp}



%Testing Brand Elements
%() This is a logo for an app. What do you think the app is for?
%() There's an app called Hunchat. What do you t

%Communication
\section{Competitor's Communication}\label{comp}
\end{document}
