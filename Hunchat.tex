% !BIB TS-program = biber

\documentclass[11pt]{article}
% \usepackage{lipsum}
\usepackage[margin=1in,includefoot]{geometry}
\usepackage[backend=biber, style=apa]{biblatex}
\usepackage{setspace}
\usepackage{mathptmx}

\addbibresource{/Users/tiagomarques/Desktop/hunchat/hunchat.bib}
 \onehalfspacing
 \setlength{\parskip}{1em}

\begin {document}

%Title page
\begin{titlepage}
\title{Integrated Marketing Communication Plan for Hunchat}
\author{Tiago Marques}
\maketitle
\end{titlepage}

\pagenumbering{roman}

%Summary
\section*{Summary}
\addcontentsline{toc}{section}{\numberline{}Summary}
This dissertation is an Integrated Marketing Communication (IMC) plan for the newcomer social media app called Hunchat.
  \par
 \textbf{Keywords} IMC, Hunchat, Social Media
\cleardoublepage

\thispagestyle{empty}

%Table of Contents
\tableofcontents
\cleardoublepage 
\listoffigures
\listoftables
\thispagestyle{empty}
\cleardoublepage 
\setcounter{page}{1}
\pagenumbering{arabic}
 
%Introduction
\section{Introdutction}\label{intro}

\newpage	
%Review of Literature
\section{Review of Literature}\label{review}
\subsection{Integrated Marketing Communications}
The last to decades of the twentieth century have brought several innovations on the way companies were managed. The exacerbation of capitalism, deregulation of markets in the western nations due to the increasing prevalence of economic liberalism have created a highly competitive environment that required more creativity on the strategic side of  business managers in order to stay relevant. Integrated Marketing Communication is one of those innovations whose beginnings go back to the eighties but whose mainstream popularity only came in the nineties and has ever since took over marketing communication decisions. In a world where advertising was the center of all marketing practice, IMC plans were surely a competitive advantage for its users. Its astonishing results came to break the ruling paradigm - in 1985 the great majority  of companies' marketing budget (75\%) was spent on advertising, that number decreased to 25\% in 2005 - meaning along those 20 years resources were gradually more widely spread across all communication channels (\cite{holm}).  
 
Ever since multiple definitions for the term have surfaced. Some simplifying it to the mere management and control of all market communications, others defending it also implies ensuring the brand positioning, personality and message are all congruent, and some even adding its need to be the most efficient (best use of resources), economical (at minimum cost), and effective (maximum results) strategy (\cite{smith}). The main ideia behind it is - as the name suggests - a consistent delivery of a chosen message across all market communications or as  \citeauthor{kitchen} put it in \citeyear{kitchen} the goal is to create a 'one-voice' brand phenomenon (p. 19). That can occur at one or more of the different levels of integration: (1) vertical - the assimilation of marketing goals and corporate goals; (2) horizontal - marketing communications aligned with management functions; (3) marketing mix - having the mix price, product and place follow promotion decisions; (4) communications mix - tools guiding customers consistently; (5) creative design - executed accordingly to the product positioning; (6) internal and external - assure all departments and employed agencies work based on an equal plan and strategy; and (7) financial - the use of the budget as most effective and efficient as possible (\cite{smith}). 

The beginning of the second millennium reinvigorated IMC with a more strategical approach instead of the widespread tactical one. Instead of focusing on short-term goals, companies began to grasp the perks the competitive advantage given by thinking long-term. Organisations were now more focused on what was their identity, profile and overall image and made sure the scope of organisation's activities reflected what their owners wanted them to be like (\cite{holm}).

Just as three notes can compose some songs but twelve notes can make compose any song, the rise of new media means more and different opportunities to communicate. The rise shifted the emphasis from unilateral mass communication to bilateral targeted communication. What do such drastic changes on the mediatic landscape mean for IMC? Is integrated communications still important in this new world?
 
 \subsection{IMC in the Digital Age}
 
 What once was an advantage may have becoma a disadvantage. The rise of a new diverse collection of IMC options created a lot of new problems on the integration of messages furthering the misalignment between strategy and tactics (\cite{holm}.) The huge impact of the digital revolution on culture and society has manifested mainly by the way we communicate. The creation of social media completely broke the ruling paradigms of ... No one could ever predict the groundbreaking network based media now ubiquitous (or at least at this extent), but awareness of the power of technological advancements were quite common on researching IMC. Some expected the development of a media anarchy (\cite{solomon}), others that it it would need to account user generated content (\cite{ananda}). The core model regarding the more tactical and strategic issues (such as \citeauthor{schultz} in \citeyear{schultz} or \citeauthor{duncan} in \citeyear{duncan}) although designed with traditional media in mind still applies to modern communication channels such as social media for their conceptual frameworks do not raise any significant barriers in its implementation. That is not the case for much of the traditional IMC models being developed when most media interactions were outsourced since they do not consider the implications of message delivery in so many, volatile, and widely specific channels. As \citeauthor{mcluhan} famously pointed out "the medium is the message" (\citeyear{mcluhan}), and with so many different media each with its unique specificity (history, context, surrounding communities, mediatic traits...) it is quite challenging (if not impossible) to encompass the same core message across all of them. The problem is no longer only which message to tell, but also which medium suits it best. Not only is this wider amount of touch points threatening, but also their nature. Direct messaging, user generated content like comments, publications... shift the control over the message from the organisation towards the consumers, only aggravating the already difficult task of communicating consistently. Social Media also introduced a blurring of marketing functions into one - sales, promotion and services are now mushed together and sometimes indistinguishable from each other (\cite{valos}).  
 
 While raising all these of implementation problems for IMC, social media has also brought an infinite amount of new and different opportunities to communicate. The real time feedback from consumers, the ability to better monitor the performance of campaigns and its impacts, the way larger amount (probably too much) of analytics and insights collected from consumers can all strongly contribute to better and more informed IMC decision making. It also forms a much more efficient, effective and economical way of communicating since its delivered only at the targeted people - highly reducing the churn rate - for the desired time, at the best context with as much specificity wanted for a small a parcel of the cost of the traditional media. The engagement from consumers provides an ongoing interaction, giving more agility and possibilities of integrations. If there's a party greatly winning for its presence on social media it is the advertisers. The abandonment of a sporadic one way communication should not be seen not as a threat to IMC, but as an improvement on its ability to form a continuous, transparent and free dialogue between customers and organisations.
 
 
 
 In \citeyear{valos}, \citeauthor{valos} developed a decision-making framework that integrates social media in the IMC process. 
 
 
 
 
Since the advent of such practices, communication has only become even more important within the context of corporations since its impact over stakeholders has greatly increased. Technology has made communication not only crucial for the human counterparts but has also to its artificial ones - with many of market decisions being made for algorithms (at an ever faster speeds).  	

 \subsection{Frameworks}
 The long 40 years of IMC existence and increasing popularity made it evolve 
 	
		

\newpage		
%Method
\section{Method}\label{method}
...

\cleardoublepage
\newpage	

%References
\addcontentsline{toc}{section}{\numberline{}References}
\printbibliography

\end{document}
