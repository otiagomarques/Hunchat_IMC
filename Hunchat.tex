% !BIB TS-program = biber

\documentclass[12pt]{article}
\usepackage[margin=3cm,includefoot]{geometry}
\usepackage[backend=biber, style=apa]{biblatex}
\usepackage{setspace}
\usepackage{multirow}
\usepackage{mathptmx}
\usepackage{fancyhdr}
\usepackage{array}

\newcolumntype{P}[1]{>{\raggedright\arraybackslash}p{#1}}

%Header
\pagestyle{fancy}
\fancyhf{}
\rhead{An IMC Plan for Hunchat}
\lhead{Tiago Marques}
\fancyfoot[C]{\thepage}

\addbibresource{/Users/tiagomarques/Desktop/hunchat/hunchat.bib}
 \onehalfspacing
 \setlength{\parskip}{1em}

\begin {document} 

%Title page
\begin{titlepage}
\title{AN INTEGRATED MARKETING COMMUNICATION PLAN FOR HUNCHAT}
\author{by Tiago Marques}
\maketitle
\end{titlepage}

\pagenumbering{roman}

%Abstract 
\section*{Abstract}\label{abst}
\addcontentsline{toc}{section}{\numberline{}Abstract}
This dissertation is an Integrated Marketing Communication (IMC) plan for the newcomer social media app called Hunchat.
  \par
 \textbf{Keywords} IMC, Hunchat, Social Media
\cleardoublepage

%Acknowledgments
\section*{Acknowledgments}
\addcontentsline{toc}{section}{\numberline{}Acknowledgments}

For their help on making this project possible I am extremely grateful.

To Patricia Tavares for convincing me to not give up; for supervising the whole process; bu specially to put up with my cynical and critical positions on every single minute detail

To Neli for introducing me to Hunchat.

To Zé and Ernesto for not conforming with the status quo and believing in a better online communication landscape for meaningful conversations.

To my uncle and aunt, Morais and Locas, for giving me a where I could to sleep.

To my loving father for financing the whole thing and did not disown me for my raging rants about the current state of world.
\cleardoublepage

\thispagestyle{empty}



%Contents
\tableofcontents
\cleardoublepage 
\listoffigures
\listoftables 
\thispagestyle{empty}
\cleardoublepage 
\setcounter{page}{1}
\pagenumbering{arabic}
 
%Introduction
\section{Introdutction}\label{intro}

\newpage	
%Literature Review
\section{Literature Review}\label{review}
\subsection{A Brief History of Integrated Marketing Communications (IMC)}\label{IMC}
The last two decades of the twentieth century have brought several innovations on the way companies were managed. A highly competitive environment required more creativity on the strategic side of business managers in order to stay relevant. Integrated Marketing Communication (IMC) is one of those innovations whose beginnings go back to the eighties, but whose mainstream popularity only came in the nineties and has ever since influenced over marketing communication decisions. In a world where advertising was the center of all marketing practice, IMC plans were surely a competitive advantage for its users. Its results came to break the ruling paradigm - in 1985 the great majority  of companies' marketing budget (75\%) was spent on advertising. That number decreased to 25\% in 2005 - meaning along those 20 years resources were gradually more widely spread across all communication channels (\cite{holm}).  

Ever since multiple definitions for the term have surfaced. Some simplifying it to the mere management and control of all market communications, others defending it also implies ensuring the brand positioning, personality and message are all congruent, and some even adding its need to be the most efficient (the best use of resources), economical (at minimum cost), and effective (achieving maximum results) strategy (\cite{smith}). The main idea behind it is - as the name suggests - a consistent delivery of a chosen message across all market communications or as  \citeauthor{kitchen} (\citeyear{kitchen}) argued - the goal is to create a "'one-voice' brand phenomenon" (p. 19) that can occur at one or more of the different levels of integration: (1) vertical - the assimilation of marketing goals and corporate goals; (2) horizontal - marketing communications aligned with management functions; (3) marketing mix - having the mix (price, product and place) following promotion decisions; (4) communications mix - tools guiding customers consistently; (5) creative design - executed accordingly to the product positioning; (6) internal and external - assure all departments and employed agencies work based on an equal plan and strategy; and (7) financial - the use of the budget as most effective and efficient as possible (\cite{smith}). 

The beginning of the second millennium reinvigorated IMC with a more strategical approach instead of the widespread tactical one. Instead of focusing on short-term goals, companies began to grasp the perks of the competitive advantage given by thinking long-term. Organisations were now more focused on what was their identity, profile and overall image and made sure the scope of organisation's activities reflected what their owners wanted them to be like (\cite{holm}). Ever since the development of a wide new range of technological advancements - mainly drawn by the widespread use of the internet - gave rise to brand new way of communicating which impacted almost every aspect of human civilization meant the entire marketing communications system was no longer tilted in favour of the seller (\cite{kliatchko}). The fabric of communication was changing and the number of new media growing - computers, smartphones, tablets, search engines, social media, online forums...  Just as three notes can compose some songs but twelve notes can make compose any song, the rise of new media means more and different opportunities to communicate. The emphasis shifted from unilateral mass communication to bilateral targeted communication. What do such drastic changes on the mediatic landscape mean for IMC? Is integrated communication still important in this new world?

 \subsection{IMC in the Digital Age}\label{digitalage}
 
What once was an advantage may have become a disadvantage. The rise of a new diverse collection of IMC options created a lot of new problems on the integration of messages, furthering the misalignment between strategy and tactics (\cite{holm}.) The huge impact of the digital revolution on culture and society has manifested mainly by the way we communicate. The creation of social media completely changed the ruling paradigms of mass communication. No one could ever predict the groundbreaking network based media that is now so ubiquitous (or at least at this extent), but awareness of the power of technological advancements were quite common on researching IMC. Some expected the development of a media anarchy (\cite{solomon}), others that it would need to account user generated content (\cite{ananda}). Nevertheless the core model regarding the more tactical and strategic issues (such as \citeauthor{schultz} in \citeyear{schultz} or \citeauthor{duncan} in \citeyear{duncan}) although designed with traditional media in mind still applies to modern communication channels such as social media for their conceptual frameworks do not raise any significant barriers in its implementation. That is not the case for much of the traditional IMC models being developed when most media interactions were outsourced since they do not consider the implications of message delivery in so many, volatile, and widely specific channels. As \citeauthor{mcluhan} famously pointed out "the medium is the message" (\citeyear{mcluhan}), and with so many different media each with its unique specificity (history, context, surrounding communities, mediatic traits...) it is quite challenging (if not impossible) to encompass the same core message across all of them. The problem is no longer only which message to tell, but also which medium suits it best. It isn't only this larger amount of touch points that's threatening, but also their nature. Direct messaging and user generated content like comments, publications, stories... shift or at least balance the control over the message from the organisation towards the consumers, only aggravating the already difficult task of communicating consistently. Social Media also introduced a blurring of marketing functions into one - sales, promotion and services are now mushed together and sometimes indistinguishable from each other (\cite{valos}).  

While raising all of these implementation problems for IMC, social media has also brought an infinite amount of new and different opportunities to communicate. The real time feedback from consumers, the ability to better monitor the performance of campaigns and its impacts, the larger amount of analytics and insights collected from consumers can all strongly contribute to a better and more informed IMC decision making. It also forms a much more efficient, effective and economical way of communicating since it's delivered only at the targeted people - highly reducing the churn rate - for the desired amount of time, at the best context with as much specificity wanted for a small parcel of the cost of traditional media. The engagement from consumers provides an ongoing interaction, giving more agility and possibilities of integration. If there's a party winning for its presence on social media it is the advertisers. The abandonment of a sporadic one way communication should be seen not as a threat to IMC, but as an improvement on its ability to form a continuous, transparent and free dialogue between customers and organisations (\cite{ananda}). 
 
It is also important to point out that modern communication on social media is usually mediated by intelligent black box algorithms by which many of market decisions are made - whether it is by high frequency stock trading (\cite{hft}) or personal feed curation (\cite{algo}). What this means for IMC is that it should account for these new artificial 'readers' of content as to achieve the maximum exposure, considering its possible effects on all stakeholders without jeopardising the original message.  

 \subsection{The Importance of Communication for Social Networks}\label{SN}

A Social Network is not only a medium of communication but it also needs to be communicated itself. The market of social media apps is quite large - there are fifty thousand new competitors every month (on the app store) all trying to be the next facebook (SOURCE) - and so standing out from the huge crowd is a challenge and communication is a crucial help.
 
Analysing the narratives that big players in the industry have used to rise to the top there is a clear commonality: counter-culture. That is not unique for social networks but for most tech companies whose beginnings were viewed as the ultimate hope for social change. The fact that digital technology seemed free of human fragilities, it promised to solve most human made societal problems. Even though the hopeful prospects for tech might have not fully realised, all major social networks still have each broken some ruling paradigm - instagram revolutionised the way pictures are shared online, facebook changed the way we share our personal information on the web, whatsapp took online chatting and made it personal - and most made sure to communicate that revolution (\cite{evil}).
 
What really makes communicating a social network so different from any other business is the importance of reaching the right people. The product sold is not only the tool itself but mainly the people who use it (hence the name 'social network') - one does choose which social media app to use mostly based on its participants. That not only implies a necessity of a minimum amount of people using the product, but also a concern for which people exactly. A common strategy to solve the participation problem is to ensure exclusivity. Facebook has achieved it by exclusively accepting new users who were directly invited by those already using it (imitating universities' social clubs) (\cite{zucked}). This strategy not only ensures that new users already have other users to connect with (which is the main goal of a social network), but can also serve as a form of buzz marketing - one must always wonder what's happening behind closed gates. The success of clubhouse has just proven how effective opting for this path can still be.
 
After reaching a solid base of users the next big challenge in communicating a social network is to stay relevant.  There are plenty of social networks who manage the difficult task of reaching a large amount of users but then lack in means to keep them interested, in fact 80\% of all apps get deleted within the next three months.

For the ones who do overcome all these hurdles it then arises the bigger struggle. As it has been shown by almost every social network achieving this long-term widespread popularity, the big tech monopolies always want a piece of the pie and will try to acquire it. That was the case for WhatsApp, Instagram (bought by Facebook), Youtube (bought by Google), LinkedIn, Discord (bought by Microsoft), Tumblr (bought by Yahoo) or Twitch (bought by Amazon). The ones who are brave enough to resist and persist on continuing on their own usually end up being copied and then have the difficult task of competing with their own product with the added hassle of the competitor's ownership of a larger, more engaged, and already fully developed network of people. That was the case for TikTok, Snapchat (copied by Facebook) . Then there are the small few cases who despite this arduous environment still manage to retain its network. That is the case of Twitter and Reddit.

All this raises three questions: (1) How can a social network differentiate from the big ones as a newcomer?; (2) How can a social network stay relevant after reaching a large audience?; (3) How can a social network stand out even when larger competitors are able to use the tools that differentiated them in the first place?; All questions in which communication can be a crucial part of the solution. Therefore such hazardous challenges require the use an IMC conceptual framework capable of addressing all of them.

 \subsection{IMC Frameworks}\label{fw}
 
The ever changing nature of technology whose velocity has been specially fast in the last two decades created a morphing landscape more fit to those who adapted their strategies accordingly.  That adaptation can be seen through the evolution of IMC. The almost forty long years of its existence made it gradually evolve to ever more complete frameworks. We have come a long way since the view of \citeauthor{nowak} (\citeyear{nowak}) passing through the \citeauthor{kliatchko}'s people based view framework (\citeyear{kliatchko}) and \citeauthor{valos}  (\citeyear{valos}) already devised a decision making framework that integrates social media in the IMC process. 
 
  \subsection{Conclusions}\label{conclusions}
  
  Integrated Marketing Communications started at the end of the twentieth century with the goal of integrating messages across all different media. Ever since it has morphed according to the mediatic changes brought by technological progress of the communication landscape. Social media and the current way of communication poses a threat for IMC - integrating a message across so many different and specific media with the aggravation of having way less control over what’s being communicated is extremely difficult - but also brings new opportunities to communicate - real time feedback, better monitoring capabilities, more consumer insights - more efficiently, effectively and economically. The Social Network market is extremely competitive which makes it hard to stand out. Counter-culture is a common narrative for newcomer social networks. The users make the product and exclusivity is a way to solve the participation problem. Staying relevant is the next challenge. Being bought or copied by the big tech monopolies follows. The goal goes: differentiate yourself, stay relevant, compete with your own product. IMC frameworks evolved with the landscape - some are purely conceptual, some integrate specific (like social media). 
 


%Conceptual Framework
\section{Conceptual Framework}

The superiority \citeauthor{clow}'s framework for IMC is clear - not only for its complete and robust body of contextual analysis and practical application tools, but also for its strategic versatility (and more abstract nature) enabling an easy integration of social media. Even though this framework excels in assuring an overall communication integration, it lacks in considering how the use of brand elements in accordance with the key message might be crucial to a full integrated marketing communication (and specifically on this context as seen on \ref{SN}). For that reason I will encompass \citeauthor{kliatchko}'s brand and consumer auditing phases from the framework he proposed in \citeyear{kliatchko}. The conceptual framework for the IMC plan for Hunchat will then be as represented in table \ref{table:concept}. 

\begin{table}  [htbp]
\small
\centering
\caption{Conceptual Framework}
\label{table:concept}
\begin{tabular}{ll}
Author             &  Plan Phase(s)    \\ 
\hline
\cite{kliatchko} & \begin{tabular}[c]{@{}l@{}}(1) Consumer Auditing;\\ (2) Brand Auditing\end{tabular}                                                                                                                                                                                         \\ \hline
 \cite{clow}  & \begin{tabular}[c]{@{}l@{}}(3) Internal Analysis; \\ (4) External Analysis; \\ (5) SWOT Analysis; \\ (6) Goals; \\ (7) Strategy; \\ (8) Messages; \\ (9) Tactical Plan;\\ (10) Media Plan;\\ (11) Implementation;\\ (12) Budget;\\ (13) Evaluation and Control\end{tabular} \\ \hline                                
\end{tabular}          
\end{table}

%Method
\section{Method}\label{method}
The research is based on both primary (exercised for this purpose) and secondary (already available) data collection. 

\subsection{Primary Data Collection}
The primary data is collected by: (1) a structured interview with each founder of the company (\ref{ze}, and \ref{ernesto}); (2) an unstructured interview with both founders(\ref{disc}); (3) three long form interviews with members of the current waiting list of users eager to try Hunchat (\ref{fwl}); and (4) a survey for the general public (\ref{qgp}). 

\subsubsection{Structured Interview with each founder}
An interview was done with each of the two founders seeking to answer questions about all the different phases of the plan. It is part of the qualitative research done in order to serve as guide for the final IMC plan. The questions were constructed as to respond to all the needs for information for each phase as Table \ref{table:founder} shows. The interviews itself were semistructured, meaning the script was open for changes during its course (adding or removing question, switching their order...) which allows it to morph to the desired direction. Both interviews were done via video call and had approximately half an hour in duration. The full transcripts are available on the appendix - on \ref{ze} for José and on \ref{ernesto} for Ernesto. 



\begin{table}[htbp]
\small
\caption{Questions for Founders}
\label{table:founder}
\centering
\begin{tabular}{ @{}P{0.15\textwidth}P{0.2\textwidth}P{0.6\textwidth}@{} }
Author                      & Plan Phase             & Question                                                                                                                                                                                                                                                                                                                                                                                                                         \\ \hline
  &  (1) Context & (1)Why did you create Hunchat? \par (2)Can you tell its story? \\ 
\cite{kliatchko}             & (2) Brand Auditing     & (3) Describe the product.What are its benefits and functions? \par (4) What need does Hunchat satisfy? \par (5) What is different about Hunchat comparing with competition? \par (6) What are the brand values?   \par (7) How is Hunchat distributed? Is it the same as competition?   \par (8) Define Hunchat with 3 words.   \par (9) What does Hunchat mean to you?    \par (10) Why are you helping building it? 
\\
 \hline
 		       & (3) Internal Analysis; &  (11) How are you currently communicating and why?   \par  (12) What is the revenue model for Hunchat?   \par  (13) How and how much is it priced?                                                                                                                                                                                                                                                \\
 \cite{clow}                           & (4) External Analysis; & (14)Who are Hunchat’s direct competitors?   \par  (15)What market are you in?   \par  (16) Identify the pressure groups for Hunchat.   \par (17) What do incubation programs mean for the future of Hunchat?                                                                                                                                                                                             \\
                            & (5) SWOT Analysis;     & (18) What are Hunchat's strengths?   \par  (19) What are its weakness?  (20) Where do you see opportunities?   \par  (21) What might threaten Hunchat's success?                                                                                                                                                                                                                                 \\
                            & (6) Goals;             & (22) What are your goals concerning communications?                                                                                                                                                                                                                                                                                                                                                                              \\
                            & (7) Strategy;          & (23)What is the segment you want to reach? \par (24) What is Hunchat’s positioning?  \par  (25) Who is the target audience?                                                                                                                                                                                                                                                                      \\
                            & (8) Messages;          & (26) What do you want to communicate about Hunchat?                                                                                                                                                                                                                                                                                                                                                                              \\
                            & (10) Media Plan        & (27) Where is it mandatory for the brand Hunchat to be?                                                                                                                                                                                                                                                                                                                                                                          \\
                            & (11) Budget            & (28) What is Hunchat's communications budget?                                                                                                                                                                                                                                                                                                                                                                                    \\ 
\hline
\end{tabular}
\end{table}

Although there's a clear vision and mission behind Hunchat, the path to get there is still fully open. It became clear that the ultimate communication goal is to explain what is Hunchat. Zé seems more inclined towards a global view when he talks about his desire for new people to meet and start great things over Hunchat - “A story I like to tell myself is that someone will meet someone on Hunchat and start a business, just as we did on twitter. I want that other people are able to do it in a platform that we built.” Ernesto is more inclined towards a more local view mentioning how he wishes for people to connect with their friends and family over the app. Both agree that their mission is to “be the best place to have conversations online”. To solve those incongruences I had to join both of them to discuss the future of Hunchat.

\subsubsection{Unstructured Interview with both Founders}
In this unstructured interview the main goal was to align both founders ideas about Hunchat and understand exactly what their mission is, why Hunchat exists and how they are planning achieve their business goals. 

It became clear form this discussion that Hunchat wants to be a new communication medium with his own specificities. Not necessarily competing with alternative media but coexisting with them - such has photography did not threat the existence of painting, of cinema the existence of photography. Hunchat presents a new dynamic for changing ideas online and that's enough of a difference to stand out. It also became clear they want to focus on the positive sensations Hunchat can bring.

\subsubsection{Interviews with the Waiting List}
The second part of the qualitative research are personal semi structured long form interviews  with three signers of the waiting list of people eager to use Hunchat. The fact that these people are already familiar the brand, like it enough to subscribe to their newsletters, and offer their time to test the app as soon as it out make them ideal to (1) audit the brand and the consumer. The questions were therefore constructed mainly for that purpose. It can also can add some research on their (2) social media use habits, and understand (3) which messages they seem more prone to receive and (4) in which channels should they be communicated. Table \ref{table:wl} relates the questions to the different phases of the plan.

\begin{table}[htbp]
\small
\caption{Questions for the Waiting List}
\label{table:wl}
\centering
\begin{tabular}{ @{}P{0.15\textwidth}P{0.2\textwidth}P{0.6\textwidth}@{} }
Author                             & Plan Phase             & Question                                                                                                                                                                                                                                                                                                                                                      \\ \hline
\multirow{2}{*}{\cite{kliatchko}} & (1) Consumer Auditing  & (1) How did you get to know Hunchat? \par (2) Why is Hunchat different?\par (3) What social media do you use?\par (4) What do you use Social Media for? \par (5) Do you discuss your ideas online? \par (6) Do you communicate using video regularly? \par (7) Do you feel the need for better communication apps?                                                                                                                  \\
                                   & (2) Brand Auditing     & (8) Describe the product. What ate its benefits and functions?\par (9) What need does Hunchat satisfy?\par (10) Define Hunchat with three words.\par  (11) What does Hunchat mean to you?\par (12) Why did you join the waiting list?\par (13) What is different about Hunchat comparing with competition?\par (14) What are Hunchat's brand values? \\ \hline
\cite{clow}    & (4) External Analysis; & (15) Do you have meaningful discussions online? \par (16) Where or why not? \par (18) What's good and what is missing on current social media?                                                                                                                                                                                                                                                                                                               \\
	& (8) Messages &  (19) What do you think Hunchat should communicate? \par (20) Are you concerned about privacy issues? \par (21) Do you feel the content you usually consume is meaningless? \par (22) Do you feel you cannot fully express your ideas on existing social media apps? \par (23) Do you feel synchronous video calls are inconvenient?  \\
                                   & (10) Media Plan        & (24) Which channels seem more fit to learn about Hunchat?                                                                                                                                                                                                                                                                                                       \\ \hline
\end{tabular}
\end{table}

The interviews were done and recorded over zoom and had an average duration of an hour. The interviews were then transcribed and analysed. Selected excerpts are on the appendix (\ref{fwl}).


\subsubsection{Survey for the General Public}
A survey to the general public can tell us (1) if there's really a need for an asynchronous video social network; (2) what people look for in a new social media; (3) from which channels can we reach them; (4) which potential narratives resonate better. 

The survey was published on August 15 and withdrawn on September 13 amounting to a total of 46 responses. Before release a pre-test was done with five participants to find any gaps in communication, any technical problems in the building of the survey and find an estimated a time for answering. All the problems were then solved (spelling, better conditions...). The link for the survey was shared to m personal group of friends and acquaintances (making it a non-probabilistic sample by convenience) and was then shared by some of them. Making this a not representative sample. The full form can be retrieved on the appendix (\ref{qgp}).

Analysing the results and considering that almost two in three people admit not being able to fully express on existing social media apps often or more (64\%, N=46), seek an ad-free social network (63\%, N=46), and claim to be interested in a social media app for asynchronous video conversation (65\%, N=46), there's strong evidence that (1) there's actually a need for this social network. Most need it because of the easiness that the video format entails, others because of the novelty factor, some because of the now lost problem of scheduling or the usual technical problems of live video calls, and other just because it seems a more prone format to incite meaningful conversations. Those who are not interested either feel that this can already be done on existing social media (since you can already easily send video messages) and find it limiting, or just aren't active in this type of channels.

(2) What do users want it this new app? Concerning the nature of the conversations private conversations got the spotlight, although almost an half want both private and public conversation and only 5\% want it to be exclusively public. The topics to be discussed there are almost equally distributed - almost everyone seems to be equally interested in intellectual, personal, or humour related interactions. Seven in ten would not be interested in the app if they had to pay for exclusive features (73\%, N=30). Half claim to dislike influencer culture (49\%, N=46), and the majority that current social media is too commercial (76\%, N=46).

The goal of Hunchat is helping people having more meaningful discussions online, but in fact the large majority of people is already having them (85\%, N=46). Where? Mostly on WhatsApp and Instagram (39\% and 27\% respectively) but also on Facebook, Discord, Reddit and Youtube (10\%, 6\%, 4\% and 4\% respectively). Youtube videos is the most consumed medium with 19\% of the inquired population watching them on a regular basis, followed by television with 14\%, Online Press with 13\%, TikToks and podcasts with 10\% each, radio with 8\%, and just after come all the social media (Twiiter, Reddit, Discord, Twitch, instagram and blogs) all with less than 6\% each. Which means that (3) YouTube, Television, Online Press, TikTok and Podcasts are the channels that reach the most people.

Regarding narratives (4), the problem of scheduling a conversation is not that common since in every four people only one claims that it happens quite often while for the others it never or rarely happens. And when asked about video calls specifically more than half never or rarely consider them inconvenient. Meaning this is not the most relatable narrative to centre the communication on. 90\% of inquired agree that the content they're consuming is sometimes and often meaningless, but 75\% admit having meaningful discussions online sometimes and often. 78\% are concerned about privacy issues, which makes telling them that the app does not collect their data important, as it is to show them that there are no ads, since 63\% seek an ad free social network. Half are interested in further discussing their ideas online. 

\subsection{Secondary Data Collection}
The secondary data is collected from the current communication of Hunchat and its competitors  (\ref{comp}) - Twitter, Instagram, Reddit, and Youtube history of content shared in their official social media pages. The data was then analysed as to spot the common trends in the Hunchat's and its competitors communication online. 

In the instagram's instagram page we can see how they share an highlights of its user generated content, partner with some of its creators to produce original content, adhere to trending movements and campaigns (such as pride month, latinx month, the covid vaccination campaign or black lives matter). They also introduce all heir new features as they come out. Youtube also partners with their creators and actively gets them togheter to enhance youtube community related topics (the rise of the creator economy, race to equality, youtube rewind...) on their own channel. They also adhere to the trending movements and produce original content on those topics (black history month, mental health day, pride month...). They also put a recurring emphasis on going independent as creator, which is in accordance with the counter culture messages common of tech companies. By going independent you're going against the grain - you no longer need the music industry, big television companies, to be discovered by scouts, you can just put it directly on YouTube. Reddit's own sub is also filled with creator's content which also get shared on their other social media official pages in little more digestible compilations. They also announce AMAs (a common culture on reddit where usually an expert lets anyone 'Ask Me Anything'. They also introduce new features as does Twitter. Twitter relies more on humour, sharing more sporadically than its average user (once a week or less) short funny tweets about the social network  and sometimes even retweet some of the most popular tweets about the social network itself. They also enter on trends such as making NFTs, responding to most requested features, announcing all the littlest design changes in a simple, clear and humorous way.


\subsection{Conclusions}\label{meth:conclusions}
From the analysis of the collected data from the interviews and discussion with the founders we can: extract all the information from the company for all the different parts of the plan; conclude that Hunchat's goal is be fittest medium for meaningful conversations online; communication should be focused on the fun and positive sensations Hunchat can bring. From the interviews with waiting list could found that consumers: are aware of the product and its benefits, the core values of the brand and what makes it different; they are used to communicate their ideas online but find that video has a high barrier of entry and are not comfortable using it regularly; are not concerned about privacy issues, are enthusiasts of specific areas, seek better alternatives than video calls, want to hear about the benefits of video; are active users of apps like Instagram, WhatsApp and YouTube, would like to have people they admire using the app. From the survey that people feel the need for an asynchronous video conversation app; do not want ads on the app, want their discussions to either be just private or have both options (private and public), want to have intellectual, personal and comedic conversations, are not interested in paid features; are having meaningful exchanges on WhatsApp and Instagram, consume mostly YouTube but also watch television, read online press, scroll trough TikTok and listen to podcasts; don't like influencer culture and believe current social media is too commercial, do not find scheduling video calls difficult, find themselves consuming meaningless content, are concerned about privacy issues, are interested in further discussing their ideas online.


%The Plan for Hunchat
\section{The Plan for Hunchat}\label{plan}



\subsection{Internal Analysis}

	\subsubsection{What is Hunchat?}
Hunchat is a new social media platform based on video focused on the exchange of ideas and conversations between users. Their goal is to be the best place for online conversation. The video platform uses a thread system (just as Twitter) where someone posts a video and then waits for responses form other people. All content is video and for now public.

	\subsubsection{Who are them?}
They are a team of two who casually ask help for specific tasks. Their main goal is making the best app they can so most of their resources are spent developing. Both founders are actively working on perfecting Hunchat in an no ending iterative process - one focused on all the backend development and the other on all the design and frontend user interface aspect. Building a a small startup is no easy task and therefore some help came to aid. Two different incubation programs - Startup Lisboa in Portugal, and Hook in France - decided to hop on and invest in them giving them not only monetary resources (fifty thousand euros to be exact), but also a way broader spectrum of opportunities to grow - from frequent exposure to possible new investors, access to important groups in European markets and even mentoring. This acts as a passport to the European startup ecosystem where a lot can be gained for the small price of sending investment reports and balances for five years.

	\subsubsection{How do they communicate?}
Since the company is based solely on the two founders, the current communication has been done by them. They are following a common trend on tech startups called \textit{building in public} - meaning they are constantly posting updates on the different phases of the process of making the app. That has been done by regularly writing full on blog posts describing specific improvements, who are then recycled on the other social media, including instagram, linkedin...

	\subsubsection{How are they perceived?}
Analysing the interviews done with the waiting list participants \ref{fwl}, there's a clear...

	\subsubsection{How can I get it?}
Hunchat is still on Apple's test flight meaning it can already be dowloaded by iOS users who want to help them find and fix all the bugs. Soon the full finalised app will be available on the App Store.

	\subsubsection{How did they make it?}
Everything is built in house?

	\subsubsection{How do they make money?}
For now Hunchat is totally free, and free of ads. The main goal for now is to get people using it and staying there. How to profit form will come later. The future revenue model can go in two ways: or it will be based on circular economy model for paid subscribers (meaning their watch-time will contribute to the creators they're watching and vice-versa); or a freemium model in which the app is still free but added features can be purchased in-app.

	\subsubsection{Conclusions}

\subsection{External Analysis}

To fully understand the innards of a company, we shall first understand the context it's inserted in.
	
	\subsubsection{Economical, Political and Social}
	
	It's not a surprising fact that the few giant social media companies are really, really profitable, but since their product is actually free, one might be left wondering 'how exactly do they make so much money?'. 'Advertising' one might say, but that's not really their product - it's data.
	
	The sate of affairs around social media is quite controversial topic. The centralised power of the Big Tech companies of the internet is getting enormous buzz around it. The main discussions concern around the algorithms that drive the feeds of users. Since these companies can control which information is shown, they can do it solely on their interest. The collusion between data companies and political parties on trying to rid elections and referendums were the first big events that launched this major concern overt these issues. By collecting an enormous amount of data from the users of these social media (something that was already being done for advertising interests), they could communicate directly to people they needed to - those who had not their decision made yet - and feed them the most biased information towards their side across multiple platforms for long periods of time. A brilliant marketing strategy some might say, but surely an ethically dubious one too.
	
	Since the advent of the internet
		
	\subsubsection{Institutional}
	Such democracy threatening scandals have put a lot of eyes on social media companies and their current privacy and curation of information practices. This led to governments to try to find solutions. One of them the RGDP (General Regulation on Data Protection) which makes it more difficult for these companies to abuse (at least without consent) the data of their users. Hunchat was born along with these concerns, so they committed themselves to collect the least data form their users as possible
	
	\subsubsection{Technological}
	Video is king. The popularity of TikTok, Youtube, Zoom... prove it.
	
	\subsubsection{Cultural}
	The impact of social media in modern life is greater than one might think. Most of human interaction nowadays is mediated by some digital communication tool. Our ideas, thoughts and opinions are all heavily by the content we consume, and most of those consuming decisions are now mediated by social media. How many times did you check reviews before going to a restaurant, hw many times you checked a book because someone shared it their social media. These networks have great impact in our well-being, on democracy and on society as a whole.
	
	 Since these platforms kept being improved for the goal of maintaining their users on the platform for the longest time and therefore show them more advertised content, they gradually became more addictive. Improved suggestion algorithms, infinite scroll, notification centres, likes... All of those surged in this iterative process of trying to get users to stay there. Similar to casino machines, these virtual systems play with our brain's reward system and make them intolerably addictive. 
	 
	 But there's a whole other side of social media. Where art, knowledge and any other ideas can be discussed freely independent of your race, sex, or social class. The ideal form of a public sphere just has \citeauthor{habermas} has utopically envisioned in \citeyear{habermas}. EduTubers on youtube, education tiktokers, an enormous amount of knowledge driven subreddits, communities on discord... all prove that the internet revolution did happen, and it's happening on social media.
	 
	 
	 	 
	 There's also a specific culture inside each network.
	 
	 
%\subsection{The Problems of Current Social Networks}\label{problems}
%- Impact on well-being - addiction, 
%- Privacy and Data Collection
%- Gatekeepers vs Free Speech (centralized power over communication)
%- Monetization.
%- Anti Trust effects: big tech deciding what gets to the users
%- Echo chamber from algorithm curation of info
%solutions
%- New alternatives- Nebula for youtube, direct payemnet to creators (patreon). Hunchat.	


	\subsubsection{Pressure Groups}
	(1) App Store policies; (2) Media; (3) Investors; 
	
	\subsubsection{Media}
	
	\subsubsection{Direct Competition}
	The players who compete directly with Hunchat in the same market are
	
	\subsubsection{Indirect Competition}
	Thinking globally about Hunchat, any social network that satisfies the same need of communication is competing with them. One of the founder mentioned '100mentors' as a potential competitor but more focused to the mentoring area.
	

	
\subsection{Consumer and Brand Auditing}
Branding is one the most important things when communicating any type of business. The brand equity is directly proportional to a company's success. For those exact reasons for this plan the first thing should be auditing the brand just as \citeauthor{kliatchko} proposed in his \citeyear{kliatchko}'s framework. For that we have to go directly to the consumers - how do they perceive the brand? What does the brand mean to them? Can they describe the product accurately? Do they rightly identify the brand values? That way we can check what are the communication needs - what's working and what's missing.

Analysing the interviews from the waiting list - that form the costumer base who have by now been active receivers of Hunchat's communication - and comparing with the founder's own idea of their brand we can clearly find some convergence. % {Convergencias}
But also some divergences. Specifically

All of the participants knew Ze personally, which makes sense being Hunchat still in its infancy days, but can be indicative of the little widespread communication Hunchat that has yet been done. The general understanding of what Hunchat is quite clear on the mind of these attentive future users, although the initial communication really influenced this batch of users. Most of them though about of Hunchat as a place for enthusiasts, as one put it "the social version of reddit". They all use Instagram and WhatsApp, some use Twitter, Facebook, Reddit and LinkedIn.  Interviews, podcasts, video calls are some of the ways highlighted for where one can have meaningful discussions right now online, but some admit ti not having them regularly and act solely on the consumer side on the interactions - just consuming ideas, but not sharing their own. They all seem to share a love for the video format - "it's faster and feels more personal", but actually don't use it a lot besides video calls, since the barrier that  to feel comfortable exposing yourself in this way is quite high and the tools are very limiting - video calls are planned and long and youtube videos require a lot of work and are unilateral.

 Concerning the brand 

\subsection{SWOT Analysis}
The SWOT analysis presented on table \ref{table:swot} consists firstly in identifying the companies' strengths and its weaknesses. The suggestions made by the founders crossed with the internal analysis make it an easy task. Going back to the external analysis opportunities and threats can be easily identified. 

\begin{table}[htbp]
\small
\caption{SWOT analysis}
\label{table:swot}
\centering
\begin{tabular}{ @{}P{0.48\textwidth}P{0.48\textwidth}@{} }
\hline
Strengths                             &  Wekanesses                                                                                                                                                                                                                                                                                                                                                            
\\ \hline  
(1) Small structure which means little costs \par (2) Loyal community around it \par (3) Digital media communication \par (4) Startup community around it \par 	(5) Being the first asynchronous video app for the masses \par (6) Global reach \par (7) Focused Team & (1) Lack of resources \par (2) Little experience \par (3) Lack of technical expertise \par (4) Slow growth  \par (5) Low communication budget \par (6) Newcomer status  \\                                                                                                                                                                                                                    \hline
Opportunities		&	Threats
\\ \hline
 (1) Under-explored space of video communication \par (2) General dissatisfaction with current social media platforms \par (3) Communities lacking an ideal \textit{medium} to communicate	\par (4) Widespread interest in having meaningful discussions onlinet	&  (1) Eventual similar players that might arise \par (2) Inability to keep up with the fast speed of technological innovation \par (3) Difficulty of retaining users  \par (4) Huge indirect competitors \par (5) Easily copible
 \\ \hline
\end{tabular}
\end{table}


\subsection{Goals}
The ultimate goal of the plan shall be (1) convincing people to use Hunchat - getting a loyal user-base that likes Hunchat enough to use it regularly (reach one hundred thousand users within the first month).  For that it is important to (2) communicate what exactly is the product and why it exists consistently in a basic and simple way.

\subsection{Strategy}
\subsubsection{Segment}
Segmentation is done to assure the most effective communication between parties. Understanding the receiver is the first way to ensure a message is not misunderstood. Dividing the audience into segments makes it possible to communicate using specific media for each segment according to each of their communication needs. Since the goal of Hunchat is to enable people to talk regardless of their age, gender, location or any other characteristic it is quite hard (if not impossible) to represent all traits on the segmentation.  Therefore simplification will be needed here. 

Besides that we've already seen how important is to target the right people form the beginning since being this a social network, they are part of the product. People need good content to stay engaged, and good content can only be made by interested people. Therefore choosing the right participants from the beginning is mandatory.

For their recently acquired critical thinking, their high quantity of social links, their familiarity with modern technology and specifically online communication through video, and their usual eagerness to share ideas, opinions and thoughts, young students will probably be the most crucial segment for the success of Hunchat. 

Another important segment to reach are people who are already participants on online discussion communities (such as inter-intellect, Omegle, discord...). The barrier to expose yourself online in such a personal way as video is quite high for people who are not already used to do it, for that reason people who are seem more prone to engage on a new medium like Hunchat. Users who are already comfortable sharing ideas online might act as a catalyst for those who yet aren't. 

\begin{table}[htbp]
\small
\caption{Hunchat's Segments}
\label{table:seg}
\centering
\begin{tabular}{ @{}P{0.2\textwidth}P{0.7\textwidth}@{} }
Segment	&	Characteristics	\\ \hline
(1) Students	&	Male and Female, 16-24, Single, High School or university students; No income to low income. \\
(2) Online Communities	& Male and Female, 20-30, Members of already existing online communities; From all incomes;
 \\ \hline
\end{tabular}
\end{table}

\subsubsection{Target}
The target audience is divided in the following groups: (1) Targeted Segments; (2) Investors (Current and Potential); and (3) Press.

\subsubsection{Positioning}
Hunchat wants to be seen as the best place for communicating online. As seen on the external analysis there is a clear demand for more meaningful ways for communicating online and Hunchat wants to fill that gap. Hunchat wants to be seen as a transparent, meaningful, organic and unique new social media platform. A place to discuss, to talk, to laugh... if it implies meaningful communication it should better be done trough Hunchat.


\subsection{Messages}
The messages are divided for each targeted group.


\begin{table}[htb]
\small
\caption{Messages}
\label{table:msg}
\centering
\begin{tabular}{ @{}P{0.28\textwidth}P{0.62\textwidth}@{} }
Target	&	Messages	\\ \hline
(1) Students \par 	&	(1) The place for meaningful conversations \par (2) Minimal data is collected \par (3) Fun and fulfilling experience \par  (4) Free to use \par (5) Rawest way to keep in touch with friends and family \par (6) Open environment for discussion  \par  (7) A place to connect with new people \par (8) A new medium to communicate online \\
(2) Online Communities & (9) Alternative space for the community \\
(3) Investors	&	(9) Hunchat is growing \par (10) Hunchat will be profitable \par (11) Hunchat's team is capable  \par (12) Hunchat stands out from competition \\
(4) Press	& 	(13) A new unique way to communicate \par (14) The first asynchronous video communication app \par (15) The best place to communicate meaningfully online \\
 \hline
\end{tabular}
\end{table}



\subsection{Tactical Plan}
	\subsubsection{Communicating the Service}
	To communicate the service to potential users and potential investors Hunchat should participate in all types of startup fairs and events - such as web summit, dev summit, slush, startup grind... So a the whole stand should be designed and built with Hunchat's message in mind. As it will be exposed on the next chapter most communication with users will be done online. It's easy on to fall on the mistake of communicating online since this is an online business but there are plenty of opportunities to communicate outside the screen. One of the segments being students, the second action to communicate the service should be distributing pamphlets on universities around the world. Even though the churn rate is quite high, the low cost makes it worthwhile specially when most of its receivers are prone to be interested. A launching party is the third action that can not only get the already invested community togheter but also incite them to bring new people who might be interested in watching the beginning of this new project starting to fly.
	
	
	\begin{table}[htbp]
	\small
	\caption{Service communication}
	\label{table:service}
	\centering
	\begin{tabular}{ @{}P{0.18\textwidth}P{0.18\textwidth}P{0.18\textwidth}P{0.18\textwidth}P{0.18\textwidth}@{} }
Goal	&	Target	&	Message	&	Action	&	Description	 \\ \hline
What is Hunchat	&	Its is a dog&	Insta Ad	& 	Promotional teaser
	 \\ \hline
	\end{tabular}
	\end{table}

	\subsubsection{Communicating Online} \label{online}
	
	Online communication will be the main way for Hunchat to communicate with their future users. Hunchat its based on the internet so there's a good chance their users are already there.
	
	The main proposal for online communication is a video content marketing campaign. Since the advent of targeted ads that we have a better control of who sees the content, how many times, and in what context... That said the first videos to be presented will show situations that evidence the contrast between the real world topics we could be addressing online and the vicious cycle of meaningless content we get stuck in. Just as examples: the continuous shot of girl in a war zone mindlessly scrolling on a TikTok feed mimicking the dances; or a teen on the movie theatre uninterestingly skipping instagram stories; or  . Then text will appear 'Is this what you wanna be doing?', 'Hunchat, the place for meaningful conversation'. These short 15 second video segments can capitalise on the ubiquitous feeling of meaningless content consumption and suggest Hunchat as the solution. The audience will (hopefully) relate to these attention catching and simply demonstrated situations and at least consider Hunchat as a way out. Therefore raising brand awareness and slightly unveiling the veil on what's about to come. Now that we've spiked their curiosity we can start revealing what is Hunchat is exactly - the same audience will be shown more descriptive videos about the app: showing it in use, the user interface, the fun of using it, and the types of conversations being done there. This will be appropriated vertically for Instagram stories ads, TikTok; and horizontally for YouTube video ads, Reddit sponsored posts.
	
	Another way to reach the desired target is to communicate on places that already contemplate the same target. Therefore the next action on the plan is choosing hand picked podcasts that are communicating to the same type of people who would use Hunchat and pay for the sponsored segments. Since these segments are usually done by the hosts the copy should be slightly altered for each show... Some examples of desired podcasts are: 
	
	There's a huge community of EduTubers (youtube creators focused on education) who would be glad to announce a product so enticing for the free critical discussion of ideas as Hunchat. Taking advantage of their established community seems only ideal. Again most of them have sponsored segments in which Hunchat will be present. Adding to that extensive video essays concerning different perspectives (for example exploring the impact different new media had on society along history and how Hunchat might be part of a new mediatic revolution) will be uploaded to the brand YouTube channel, giving the audience a deeper and exciting new ways of perceiving the product and its goals that runs away from the superficiality of short sponsored segments.
	
	Since the low budget, simple publications on already existing online communities such as subreddits, discord groups and any other online groups related to the free discussion of ideas (for example ...)
	
	Sharing the most popular and controversial 'hunches' (meaning the content shared on Hunchat) on Instagram might also serve as a catalyst for people to join for their eagerness to reply, also better understanding exactly the type of content that is shared there. And it might also inceltivize users to share their own 'hunches' on other social media.

	The current way of transparent 'building in public' communication on the blog and twitter should be maintained as it keeps audience engaged and constantly interacting with the brand and contributes to showing how important transparency is to the brand.
	
	The following table summarises this tactical plan for online communication.
	
	\begin{table}[htbp]
	\small
	\caption{Online communication}
	\label{table:online}
	\centering
	\begin{tabular}{ @{}P{0.12\textwidth}P{0.15\textwidth}P{0.18\textwidth}P{0.1\textwidth}P{0.35\textwidth}@{} }
Goal	&	Target	&	Message	&	Action	&	Description	 \\ \hline
Presenting Hunchat	&	Students, and Online Communities 	& Hunchat is the place... & 	Podcast Ad & Audio manifest on handpicked podcast with similar target* \\
	& 	&	& Instagram Ad & Vertical video (stories) campaign showing vicious circle of comm
 	\\ \hline
	\end{tabular}
	\end{table}
	
	
	\subsubsection{Institutional}
	
	\subsubsection{Media}
	Press Release?
	\subsubsection{Social Responsibility}
	Transparency, No data collection....
	\subsubsection{PR}
	TV shows, Podcasts, Youtube...
	\subsubsection{Parternership}
	Pay content creators to get on Hunchat, and their 'hunches' on Instagram (with the logo). More Brand awarness, eagerness to enfgae
	\subsubsection{Events}
	Web Summit...
	\subsubsection{Advertising}
	Google Adwords?? Magazine press?
	\subsubsection{Promotion}
	Merchadising?
	\subsubsection{Buzz Marketing}
	
\subsection{Media Plan}
The media plan...
\begin{table}[htbp]
\small
\caption{Media}
\label{table:media}
\centering
\begin{tabular}{ @{}P{0.2\textwidth}P{0.7\textwidth}@{} }
Medium	&	Description	\\ \hline
 (1) Online Communication	&	(1) ffhdif \\
\end{tabular}
\end{table}


\subsection{Implementation}
The implementation of this plan is dependent on human and financial resources available. A specialised team should work out the plan and achieve its goals.

\subsection{Budget}

Budgeting is based on allocating resources based on the proposed strategy and according to the communication and marketing goals defined (\cite{clow}). 

Since the defined budget is little and considering the marketing needs the 'objective and task' method seems the fittest. By using this system the allocation of money is solely based on each specific objective, giving the its users the elasticity...

\begin{table}[htb]
\small
\caption{Budget for the IMC Plan}
\label{table:budget}
\centering
\begin{tabular}{ @{}P{0.5\textwidth}P{0.2\textwidth}@{} }
Action	&	Budget	\\ \hline
 (1) Online Communication	&	3 000  \\
\end{tabular}
\end{table}


\subsection{Evaluation and Control}

Once applied the plan should be constantly scrutinised for evidences of its impacts in this iterative process (similar to the one of building the app) of persistent analysis and changes when needed. The first and most important short term marker in pursuit of evaluation of performance is the volume of new users - if there are more people joining it is because it is working. Long-term control can be done by understating the users engagement with app. (1) survey; (2) markers - user engagement, dropout*, meida mentioning,  (3) stats from online advertising, (4) collect and analyse given feedback - are .. and then alter the plan according.


\subsection{Summary}


\subsection{Conclusion and Recommendations}




\cleardoublepage
\newpage	

%References
\addcontentsline{toc}{section}{\numberline{}References}
\printbibliography
\cleardoublepage

%Appendix
\addcontentsline{toc}{section}{\numberline{}Appendices}
\appendix

%Collected Data
\section{Collected Data}\label{data}
\subsection {Interview with José} \label{ze}
The first interview was done in... with José Gomes one of the founders and the current designer of Hunchat. With his consent.

(1) Why did you create Hunchat? Can you tell its story? \textit{Hunchat started when Twitter launched a new feature called ´fleets' that is basically instastories or snapstories (whatever you wanna call them) for Twitter. I looked at it as wasted opportunity for Twitter to use video in a worthwhile way, and so I ranted about it on Twiiter and Ernesto saw it. He came to me saying we should do something about it. One day we picked up a Figma file} - referring to the app designing computer software -\textit{and started an initial draft of what later became Hunchat. Our goal is to enhance online conversations in the most human way possible. Our mission is to be the best way to have conversations online. Connect people, connect projects in the same way I did with Ernesto over Twitter. Create a good place por people to talk.} 

(2) Describe the product.What are its benefits and functions? \textit{Our product, Hunchat, is a video based social network focused on the exchange of ideias and conversations between users. As I already said, the goal is to be the best place for online conversation and we are a video platform that uses the thread system (just as Twitter) - when someone posts a video, then waits for responses from other people. All content is video, and now in the beginning, it is open and public - just as the timeline in Twitter, where people can interact with each other and create parallel conversations - but in the future there will be private chatrooms too.} Why private chatrooms? \textit{We have tested in little WhatsApp groups and such, and there is always a better dynamic when people already know each other and speak only for one another instead of speaking to te world in general. It is different sharing something with a friend than sharing on the internet with everyone. So we're first testing the concept publicly and then create that individual groups dynamic or even one-on-one.}

(3)What need does Hunchat satisfy? \textit{It satisfies the need for online communication, of interacting with people in more human way.}

(4) Knowing that there are already so many different apps trying to satisfy the same need. What is different about Hunchat comparing with competition?  \textit{Right now, there are only tow ways of communicating online: asynchronously - text messages, tweets, audio messages, video messages are appearing more but there's no platform designed specifically for it; Or synchronously - such as the call we are doing right now in which the conversation is more organic, we can see each other, it is a different interaction. We want to create a similar live interaction without the need of synchronism. No need to schedule. Happening in such an organic way as we do messages but as good as an interaction as live. } 

(5) What are the brand values? \textit{Our values are the connection between people. Transparency - our code is open source, anyone can see our code, we are committed to collect as minimal information possible, and have no plan of selling information as opposed to any other social network. We value privacy, openness, and connection between people.}

(6) How is Hunchat distributed? Is it the same as competition? \textit{Yes, the direct access will be in the app store in the next two months. First only in iOS, and only later will we think about android as a way managing resources.}

(7) Define Hunchat with 3 words. \textit{Transparent is important. Intimate. And the ideia of getting two people to start something new is also important. I don't knwow how to describe it.} Serendipity, maybe? \textit{Something like that.}

(8) What does Hunchat mean to you? \textit{For me, even though I'm repeating myself, is the online conversation between people. I meet a lot of people online - mostly on twitter, it was the case for Ernesto - and I believe it is really useful to create new platform where people can do it, that I also can do it in a different way. Simply, for there to be a better option for communication with other people online. For me it is also specially important, being my first company. Creating a project of this dimension for a personal project is of course important.}

(9) How are you currently communicating and why? \textit{Right now, we are only two people making it difficult to make a huge communication. So we are basing it on 'building in public'. Right now, our pipeline is: whatever is happening, we write a blog post about it, and then recycle the blog post in other content across other channels. So, from the blog posts comes all of the rest: Instagram posts, LinkedIn posts, tweets...} Is it all done online, then? \textit{It is all online communication, yes.}

(10) What is the revenue model for Hunchat? \textit{Being a B2C with a social network dimension, in the early life of this type of startup it is still too soon to think about monetisation. But right now what we are planning is making it freemium. A goof example of that model is Tinder. Usually people think about Tinder as a free app, but there is a big base of paying users that support the whole app, that are paying for extra tools. } What tools will they be, then? \textit{What we're going to do to make tools that people actually want is to make them free first, and then wait for the feature requests - "we want this", "we want that" -  and then we build them accordingly and the put them behind a paid plan. We are never going to take away things we already gave for free. There is already a big history of companies making it, and receive an enormous backlash from their community. So, as new features are requested we are going to get them behind a paywall.} And what kind of features will they be? Introducing paid features won't somehow disrupt  the democratic values you're trying to associate Hunchat with?\textit{They will be features that give the paid users no advantage over other users. For example being able to double speed the videos.}

(11) How and how much is it priced? \textit{They aren't defined yet, and there's nothing to gain on speculating about it now.}

(12) Who are Hunchat’s direct competitors? \textit{Direct competitors in terms of having the same product, we still have not identified any. Indirect competition are applications like '100mentors' - a small app that makes something similar to what we do for the mentoring area. Someone publishes a question and then a mentor shoots a video responding to it. And there are other niche apps appearing that also use asynchronous video. In a more general sense a big competitor would be Twitter. }

(13) What market are you in? \textit{In the market of social networks.}

(14) Identify the pressure groups for Hunchat. \textit{Simply I can't identify any.} 

(15) What do incubation programs mean for the future of Hunchat? \textit{We are in two different incubation programs. Startup Lisboa in Portugal and Hook in France, Paris. Both are remote, not only due to covid.} - referring to the global pandemic experienced in the time the interview was done that shifted most of social activities to be done remotely - \textit{Hook made a direct investment of fifty thousand euros for five por cent of the company. And keep up with the company for 5 months. We have workshops, they support us in whatever we need. And we have demo days where we present to investors in order to have a new insights. It is like a Y combinator but in Europe.} - mentioning the world's largest incubator of startups situated in the US -\textit{ It is the same model. Startup Lisboa help us in a different way. They don't invest directly in the company. They only give us access to important groups of the Portuguese and European markets, and mentoring sessions.} Are there any side \textit{None. There is a mutual interest, our only obligations is to send them investment reports later, and balances for five years. We. have total freedom.}

(16) What are Hunchat’s strength? \textit{Being a small company - being a small team our costs are few. Focus. Innovation in the asynchronous video area.}

(17) What are its weakness? \textit{Being small is also a weakness in many ways. Lack of resources slows down the process, we are not at the speed we wanted to.}

(18) Where do you see opportunities? \textit{Our opportunities are in the social space, and being the first asynchronous video for the masses.}

(19) What might threaten Hunchat’s success? \textit{Competition will surely later be one, but it won't kill us now. Not being able to be fast enough, entering too late on the market to catch people's attention and have a satisfying retention rate.} What do you mean by fast enough? \textit{In the startup world, everything is very fast. For you to have an idea, for the hook program it is mandatory (not mandatory more like a moral obligation) to publish something new every week. Every week there needs to be a new change in the product, a new advancement, something new based on people's feedback. And then track what went wrong, what we learned. It is really important that with the money we have, we have a limited amount of time without new capital and without making money, so when the time comes we need to be the best we can. For that we need to keep iterating. Specially in apps like ours usually the retention rate in the begging is awful (or if you're lucky it is good), but the goal is to keep getting bigger a percentual point at a time. Even twitter looses 70\% of its users. Periscope, Vine... all died because of this.} It then diverted to a discussion about the rise of social networks that is not really relevant for the plan.

(20) What are your goals concerning communications? \textit{Our goal right now is to communicate this... How do I put it? It is to communicate what the app is exactly. Being a product for the final consumer, and being the social network such an abstract concept it isn't easy for people to understand right away what it is, without experimenting it first hand. So our goal now is to find a basic and simple form of explaining exactly what is Hunchat.}

(21) What is the segment you want to reach? \textit{There are two different responses to this question: one for now, and one for later. A social network is as good as the content that users generate there. So there comes the problem of the chicken and the egg. If there isn't content, there won't be users. If there are no users, then there isn't content. So our goal is to be good to use it even with only a hundred people. And we're going to do it by entering in already existent online communities and convince them to use Hunchat. Enabling a closed system with an organic growth inside each community until they are satisfied and us convinced they won't leave. Only then explore a more open market.} Where are those communities? \textit{We are going to test soon with one. It is called inter-intellect. It is basically a corner of the internet full of nerds like me which like weird things and make saloons in which someone presents a theme and gives a mini-workshop that ends up in a three hour long conversation. Later, we can present ourselves more like a social network -in a more direct approach - just like in the street you can approach any person,}

I skipped the question '(22) What is Hunchat’s positioning?' since it seemed to be already answered. (23)Who is the target audience?\textit{For now, it is people that are already used to making video calls, that are already used to talk with other people online - and so our focus is on online communities,}

(25) What do you want to communicate about Hunchat? \textit{The importance of proximity when talking with people online.}

(26) Where is it mandatory for the brand Hunchat to be? \textit{Twitter, Linkedin, Instagram and blog posts.}

(27) What is Hunchat’s communications budget? \textit{Right now, we have no budget allocated for communication.}

(28) Who are the stakeholders? \textit{We are B2C directly and we have no advertising, our main stakeholders are the clients, ourselves, and our investors.}  Which will be most affected by Hunchat? \textit{People who use Hunchat and meet new people on it. A story I like to tell myself is that someone will meet someone on Hunchat and start a business, just as we did on twitter. I want that other people are able to do it in a platform that we built. } Who controls the resources? \textit{Me and Ernesto.} What motivates each stakeholder? \textit{For users it is clear. For me it is the fact that it is my own business and obviously want it to be a success. For the investors it is obviously a financial interest.}

\subsection{Interview with Ernesto} \label{ernesto}
The following interview with Ernesto Gonzales is one of the founders and the current main developer at Hunchat. With his consent.

(1) Why did you create Hunchat? Can you tell its story? \textit{The creation of Hunchat comes from the frustration that I and Ze had on social media. We feel that text usually doesn't capture the essence of what we're trying to communicate. We think that video is the best format to communicate, but it comes with the problem of synchronisation of calendars. For example, I am cuban, and my grandparents live in Cuba. They call me a lot of times over WhatsApp and I can't pick up because I am working. And the opposite is also true, I call and they can't pick up. Video messages solve that problem. I think that when we have all the components of verbal communication - body language, facial expressions, intonation - video is so much more closer from a face-to-face interaction than any other format. The ideia surged when twitter launched 'fleets' - which is basically stories for twitter - and Zé published a fleet questioning what uses fleets could have and one of them was this type of communication we're looking for on Hunchat. People talking about stuff just like on twitter, and with their friends on WhatsApp but with video messages. Just like a vlog - on one side for being public, and on the other for being a video message. From there we started building on the ideia forward.}

(2) Describe the product. What are its benefits and functions? \textit{Hunchat is a social network that allows its users to communicate with their friends and family using video messages. Its main benefit is the 'asynchronisation' of calendar, that problem is solved from the root. We no longer need to be both available at the same time, it allows the video format and can even helps manage the tiredness of video calls - nowadays there are video calls for everything. And then there will be the benefits from the content the users will get on Hunchat. But exactly what only time will tell, it is to soon to talk about it.}

(3) What need does Hunchat satisfy? \textit{The necessity of communication. Sometimes it's difficult with consumer products - they don't solve a problem, they take advantage of opportunities not yet explored. The problem only surges when we give and then take. Now we're giving, only then will we see what the problem is. It always comes back to the same ideia - creating a better communication context for family, friends and talking openly with the general public. I can't speak about a specific problem, we're working on improving the opportunity that video is, that we think has been under-explored.}

(4) What is different about Hunchat comparing with competition? \textit{The more direct competition might be TikTok. People always associate due to the video format. If we think about it through architectural standpoint they are quite similar. The content on TikTok is limited in time - one minute only per video - that has huge impact on the type of content shared there. On Hunchat there's no limit, just as on a video-call. We could even forget the fact that TikTok has the music, and dances. The time limit alone makes it absolutely different. The biggest difference is the content. They are building a global community, we are strengthening the local community. Our focus is friends and family. You upload a video - you go somewhere, you're showing where you were at,  you are sharing a moment with your friend, announcing some news, sharing your thoughts for the future - and show the same type of thing you would on a video-call with your friends or family. That's it.} But how does exactly an asynchronous video conversation work? Is it like voice messaging? \textit{It's exactly that. Just like a voice message, but in video.}

(5) What are the brand values? \textit{We are looking for meaningful communication. We are not looking for what already out there on instagram - influencer media, content for it's own sake just for followers... We are looking for the same type of core audience as on early Instagram. You sharing with your ideas with your friends.}

(6) How is Hunchat distributed? Is it the same as competition? \textit{When we talk about apps, the goal is usually the App Store for iOS, and the Google Play for Android. Right now, it still isn't available on the app store, but in the next few weeks it will be there. The app is already ready to launch, i just finished it yesterday. We were supposed to have launched it some weeks ago, but we had some minor problems that are now solved. We are now submitting it to Test Flight,} referring to apple's platform for testing beta versions of newcomer apps. {...and then App Store.}

(7) Define Hunchat with 3 words.  \textit{Friends, meaningful - the thing we're always after, and sharing.}

(8) What does Hunchat mean to you? \textit{It always comes back for me wanting to talk to my grandparents. It means an opportunity to obtain a better communication dynamic between me and my family. I have brothers in Angola, grandparents, cousins and uncles in Cuba. An app like Hunchat will make communication in cases like mine easier. }

(9) Why are YOU helping building it? \textit{Because I feel inspired by the project, and what it can become. I'm curious in what it will transform to. It is a challenge, and we like challenges.}

(10) How are you currently communicating and why? \textit{The communication might be the thing we've been having more difficulty on. Zé is a designer, I have a more technical background, so this is not our area of expertise. What we've been trying to do is: in the beginning we've gone from a communication inclined towards 'what is Hunchat'. Something like 'video based social app for asynchronous conversations' but that is highly descriptive. What we noticed was that 'asynchronous' is a word most people don't even know what it means and that it is too much of a 'what?'. Now we're trying to go from 'what' to 'why'. We're now trying to change our communication to the reasons why we are creating Hunchat, and transmit our vision. But I think we're not there yet. We've been saying that Hunchat is an app for communication with friends and family, for meaningful conversations - that type of language. It always ends up in meaningful conversations.}

(11) What is the revenue model for Hunchat? \textit{We had one already developed, but a problem raised. In this type of products, you need scale, so it isn't worth spending time thinking about that now. But our ideia is a subscription model of circular economy.  A freemium app in which paid subscribers enter in the circular economy. For example, 20\% of my time in the app I am consuming your content and the other 80\% I am consuming Mary's content. From the ten euros I paid for the subscription, two will go to you and eight will go to Mary. 20\% of those transactions go to Hunchat. While the free users can still use the app for free. We don't like data. We are trying to avoid collecting it, because we think that in general data will worsen the user experience.}

(12) Who are Hunchat’s direct competitors? \textit{People usually mention TikTok. And if we talk about ease of integration in the app, TikTok is in fact who has the easiest integration. But we have completely different visions: theirs is entertainment, ours is communication. So a better answer would be facebook for their fame of always copying new players, and resources (like engineers), and so on. There may be other competitors we haven't heard about yet, and those worry me way more.}

(13) What market are you in? \textit{Our strategy is now based on entering in different communities, one at a time. We are now entering in a community called inter-intellect - a community for sharing ideias and knowledge - that is global.  So we cannot talk about local markets, like the American market, or the European market. Let's just say the market of online users.}

(14) Identify the pressure groups for Hunchat. How do they affect you positively and negatively? \textit{It is too soon to talk about those. We started testing the app two months ago, but later we started reconstructing it in another programming language to make the performance better. We only used ten more people, all friends, it isn't even that used yet. It is too soon to discuss all that. }

(15) What do incubation programs mean for the future of Hunchat? \textit{The first one made the project real. It went from a two computer kid's project to two kids creating a startup. It is somewhat different. It helps with morale - when someone asks you 'what are you doing?' you no longer have to say 'I'm working on a project', you can say 'I am working on a startup'. What I mean with this is that you give another meaning, another weight to Hunchat. By entering in the entrepreneurs' community we met a lot of interesting people that can teach a lot of things. We are just starting and there's so much for us to learn, and that's important. The french market is very large on consumer products, their startups ecosystem is second the largest in Europe (after London if I'm not mistaken). It is huge and that's an advantage for us, being in Portugal being able to enter there. We're automatically introduced to investors who have more experience with these types of product. That's extremely important. Meeting people that have a good understanding of our market that can give us advice. Without even mentioning the money that they invested on us. Even without that part, it would still be an amazing opportunity.}

(16) What are Hunchat’s strengths? \textit{Our vision - making Hunchat the best place for people to communicate online with friends and family.}

(17) What are its weaknesses? \textit{We have a lack of resources. We do not have a lot of experience building this type of products. I'm the only programmer and run with big struggles. Building an app alone is already difficult, building a video app is one of the most difficult thing you could do.} He then gave some technical examples of hardship that despite being quite interesting do not regard this communication plan. {Our biggest weakness must be the lack of technical expertise. But with our motivation, at the end of the day we will have that experience.}

(18) Where do you see opportunities? \textit{In the under-explored space of video communication. For us to communicate with video and take full advantage of its potencial, there's no need to be synchronous. We already went from messages on a dove's paw with months waiting, to the mailman that gives them by hand with weeks waiting, until we got e-mail, and text messages, audio messages, phone calls. We think there's still so much to explore. Hunchat will be there, alongside so many other companies. We believe there's a lot of potencial with video.}

(19) What might threaten Hunchat’s success? \textit{Its success can only be threaten by its own team. In the startup world the biggest enemy is your own team. Right now, it is only me and Zé. It is important for us to accumulate small victories, in order to maintain our morale high. Having ambitious goals, but small and with short periods. Staying focused on our vision. It is extremely difficult to maintain mental sanity while working so young on a startup. We are out biggest challenge, not our competition, not the fact that we are not on Silicon Valley... us and our capability of staying focused are our biggest adversities.}

(20) What are your goals concerning communications? \textit{For now it is finding a language that can transmit our vision. Our vision is there, our mission is there. The struggle is communicating what Hunchat is trying to do, also what Hunchat is. It's like a spectrum. On one side we have the 'what', and on the the other the 'why'. We are trying to move towards the latter, but we haven't found the sweet spot yet. Something we have also noticed is that if we incline too much towards the 'why', people might understand it but then go 'you'll do all this for me, my friends, and family, but how?'. The ideia is: before people actually make the action we want them to do, they should already have an ideia of what Hunchat is. Maybe that can be done just with a screenshot of the app, maybe a video, or photo of someone using it. Only experimentation will tell. But we are on the right path. On the last few weeks we have been walking in the right direction, we've shifting our attention towards textual communication on Twitter and LinkedIn. With time we'll get there.}

(21) What is the segment you want to reach? \textit{In the long run, to be the best place for online communication we need to reach the mainstream, we need to be global. But right now we're reaching a community at a time. Our goal now is Interintellect, we're focused on the niche of their users.}

(22) What is Hunchat’s positioning? \textit{Transparency, focus on content that the users find important and meaningful. It is not for videos of the Rock dancing or for memes - that's entertainment. The important content for us is what originated social media in the first place on the millennium's beginning and that is what our friends and family have to share with us.}

(23) Who is the target audience? \textit{Right now it is a member of Interintellect.} But besides that who is the type of person you want to reach? \textit{I don't believe there's a specific type of people. For example who is the target of WhatsApp? There's none. If we're trying to create a product that allows me to talk with my grandmother then it can be technologically exigent, it needs to be easy to use. Record, upload, watch. It needs to be simple. Not targeted just for young people or old people. It is for everyone. But I'm talking for the long run, maybe within a year and a half. Now we're focused in idea sharing communities of the tech-twitter type. Now specifically Interintellect - a community of approximately ten thousand people. For the next two months they are our focus, and the only users of Hunchat. That's our market. We believe that's the way to go now, because if we try to build to everyone right away, we'll end up building for no one.} And after that? \textit{We are targeting other people who like us are building in public. From there move towards the startups ecosystem - entrepreneurs like us. A similar model to what clubhouse has been doing within the Silicon Valley elite. We are focused on 'build in public' communities - first Interintellect, then maybe startup Lisboa, then maybe a subreddit.}

(24) What do you want to communicate about Hunchat? \textit{I want to communicate that Hunchat is a space for sharing.}

(25) Where is it mandatory for the brand Hunchat to be? \textit{Interintellect. Zé has been living there, meeting upon meeting, to understand what they do, how they think, what they want. We need to know well our target. And because they are our target right now, that's what we've been doing. Hunchat has a mission but there isn't an essence yet. We are in the liquid estate. We're still evolving. It will be our users that will tell us what Hunchat is. The vision is here, the mission is here, but it's them that gonna tell us how to get there.} And after Interintellect? \textit{We don't like data in the app, so I don't believe we should advertise on other social media. It is a question of sharing our vision in other social networks - Instagram, Twitter, LinkedIn, and even word of mouth. We want the experience to be so good that people will invite their friends and family to join Hunchat. It will be always online. I cannot imagine a billboard or a physical event.}

(26) What is Hunchat’s communications budget? \textit{I can't say exactly. Two months ago we made the calculations. It is extremely difficult without having a certain amount of users to know how much you can spend acquiring a new costumer, because you can't know what's the costumer lifetime value, or the costumer acquisition cost. But I can say that our goal for the next six months is to have two people oriented for that. So for a one year's budget that would be one hundred and sixty thousand euros, but only spent on human resources - a marketeer and a designer for example - not on ads or anything.}

(27) Who are the stakeholders? Which will be most affected by Hunchat? Who controls the resources? What motivates them? \textit{Our biggest stakeholder will clearly be our users. But again it is still too soon to specify. We are with our heads outside the hole trying to listen, collect information. It depends on our luck, the path we take. We know our vision, we do not know the path to make it true yet. Even the users don't know - we don't know and they don't know. We need to discover. We also have our investors from Hook. And me and Ze. But it always end up on the users, they will tell us where to go, what they want and we will listen. Hunchat can work with feeds, chatrooms, can be more oriented for public, or more oriented to private. Think about the dichotomy of global and local. Right now I'm telling you it's more local - family and friends - but maybe users will push it more towards the global. We do not know. Our goal is to be the best place for online communication, how depends on the users. They have most of the decision-making power.}

\subsection {Discussion with Founders}\label{disc}

The following discussion comes from the need to better answer the questions from the first round of interviews mainly about the purpose of Hunchat - why they exist, what is their vision and how are they gonna achieve it. 

(1) Why does Hunchat exist? How is it going to accomplish its mission? And what is Hunchat exactly? Zé said that "\textit{Huchat exists from the need for people to talk in a more real way without the need to be online at the same time and to schedule an hour to be online. It is a way for people to communicate in the most real way through the internet. The way we are going to achieve it, at least until now, is with a mobile app that uses asynchronous video. The 'why we exist' is the connection between people.}" Ernesto did not have anything to add. (2) You now have an MVP running. How is it working? Is it what you expected? Is the content public or is it private? Zé "\textit{It is public, and it is based on people we trust.}" And is that the way to go from now on? The public route - an app where people communicate for the whole world as it is being published now? Ze said "\textit{We have been discussing that for weeks now. We have noticed the differences between our ideas concerning the app as you explained and we have decided to go towards the more local approach. That doesn't mean people will publish in the private mode, because it is going to be public. But local meaning you'll be using it to talk with people you already know - with your friends, and your family, and other people you invite to the app. Our approach is now that who uses Hunchat uses it to talk with friends, with family, and invites their crew, and less the view I mentioned before of meeting new people and discussing ideas. A more local and personal approach.}" (3) How is the content curated on Hunchat? Ze: "\textit{For now it isn't. You follow people, and the people you follow will appear on your feed. You also have an explore section just like on Instagram with content from people you don't follow yet in a pretty rudimental way in chronological order. There's no recommendation algorithm. Your feed with people you follow on chronological order, and the Discover with people you don't follow on chronological order.}" (4) How is the content moderated? Ze: \textit{"In the beginning it is going to be me and Ernesto watching the content and later we will need to hire people to do it for us. The content we don't want is nudity and hate speech. There are limitless interpretations for this, but that's it."} If the freedom approach is taken - Hunchat does not collect data, it does not have curation algorithms - there's always the risk for groups to with similar ideas to form an echo chamber that can be problematic. (5) Can you clarify which is exactly your positioning, target, and segentation? More specifically. Ze \textit{"We also discussed that. For now our ideal user is a prototype of ourselves. We are creating this app for people our age to talk with friends, something that is fun to use."} I wandered a little bit on why I scheduled this discussion and why it is important. Ernesto finally intervened "\textit{I can bring some vocabulary quite different from José's. We are now focused on vibe like WhatsApp meets Instagram in their first two years. There's no influencer culture here, there's no memes... It is literally to know people you have already met. It is a social network by definition, just like the old days of the internet. That's the vibe we're looking for. Why does it exist? Because there was a big shift from content from people to people to a seller to consumer (the influencer media). Our ultimate goal for Hunchat to be just as for you and your friends as to find like minded people, but right now we are not focused on the latter. It's much easier to later manage you and your friend, even for growth hacking is better. Now Hunchat exists for you to connect with friends and family trough video. It is an under-explored medium in our opinion. There's the synchronous world, and then the asynchronous like on instagram stories which is similar to what we are looking for. For example you go for a bike ride and film it telling how it is going. Or you graduated and want to share it. Or you've seen a beautiful sunset... that's the type of vibe we are looking for. The exact words I can't find.}" So in the last interview when you mentioned 'meaningful conversations' you meant 2more that type of conversation than discussing for example climate change or other ideas. \textit{For now that's it. It's whatsapp meets instagram. Closer to one-on-one conversation for you and your friends in a video format, with similar content distribution as instagram."} Ze added "\textit{"This is clearly where we have the biggest difference in our approach. When we started our idea was to go more for that type of discussion of ideas. We even had tags and we dropped them. We were gradually finding out that for the sake of the problem, our own sake, and for it to succeed we had to follow a way more personal approach. So that part is now on hold and we'll go the personal side.}" That said, isn't that need being already satisfied by other social networks? Ernesto "\textit{There's my specific case. My grandparents from Cuba calls me like five times a week, if I pick up two it's great. If we're not available at the same time we cannot speak. The asynchronous video format - which we believe it's best (at least until holograms are a thing) - and is under explored not just for being asynchronous but also on the way that is used. Just the design can really influence the way we use it. And there is no app like ours. Think of Instagram - the home page (feed), explore, notifications, and profile - there's nothing like that on video. You can use TikTok but no one is using it to talk with friends, and not even to discuss ideas. Our main difference right now is video duration and the way videos are organised in the app. Maybe that will lead to the answer you may be seeking when you ask that. But that's the answer I can give you now.}" Ze elaborated "\textit{"People put all video in the same bag but they forget that an Hollywood film is video, that that film that your father recorded of you in 2000 is video, and a TikTok is also video. So inside video the medium can still be different."} I get it. You are trying to create a new medium with its own specificity. Then we discussed some technical details about how the app is functioning right now to better understand what's public and what's not - everything is public, temporarily. (7) What do you think it's crucial for Hunchat to communicate right now? Ze "\textit{With a lot of words it is easy to say. Our site now says - Talk with your friends, even when they are not online, we are looking for that. A place to talk with friends in a new medium of communication where you talk with the people you know and are proximate with, a more personal way to connect with them. How do we communicate it in a simple way, I don't know.}" Ernesto \textit{"We want to transmit that idea of proximity with the people that surround you. We want to transmit what the app has to offer. It may sound cliche, we want for you to open the app and that when you've closed it you've smiled. Because you've seen a message from your mother, your friend, your granny... That's what we're seeking. We want to create moments of happiness, complicity. It's more important to tell our users what we're gonna offer than how."} What stops me from already doing that same thing right now with the existing apps - I can have asynchronous long duration video conversations on WhatsApp for example? Why would I join Hunchat instead? Ze said \textit{"In a simple way if this type of approach of communication between people wasn't needed, Twitter wouldn't exist. Anyone can write 180 characters wherever they want. The platform itself changes the way of communicating a lot. The platform changes the medium. Right now there's no place where you can publish... I'll better use Twitter. Twitter is the only place where you can publish something and then four or five friends of yours can comment and then reply to each other's comments and keep it going. Now imagine that on video. It will be way more similar to a group conversation without the need to create groups or having defined groups. It's way more spontaneous. Ok, maybe the video wasn't for you, but I filmed it and published it here, and you come and give an interesting answer. And maybe I didn't even remembered you when i published it.} So after all when you publish it, it stays open to everyone? \textit{"Yes, right now everything is open. We have now decided that for a resources management standpoint, but eventually it will be changed and be private.} Well, I'm having difficulty understanding what exactly is Hunchat. Every time I ask it seems I get a different answer. Then we discussed on how I could get on the app to better understand it. Ernesto added \textit{"I have a simple answer. Why do TVs still exist if you can watch it on you phone? Because it is not the same. In your phone it's way more uncomfortable because of the smaller screen size. But if you think about it, it's the same thing. The platform constraints will affect the type of content that is created there. You could make videos with music on Instagram and on Snapchat, but yet the Musica.ly surged because it had a difference. The way the videos were organised and how simple it was to add new music to the video. Another example, why are there people that prefer sending voice messages over making a call? Because it's asynchronous. Or twitter? Why did Twitter surge if e-mail, and messages already existed? Or Instagram, why did it exist if Photos already existed like three years earlier? The constrains of a platform and the type of content you consume affect the way you will produce and create content. We are exploring an intersection that until now wasn't used in a mainstream way. There isn't a platform with asynchronous video only where you can share and communicate with your friends all at the same time. We are talking about videos with no limit, anything can happen. And eventually we will be surprised by how people will use our app. We think of it in a way, but people might think in other ways. There are always differences. For you to make that question it's because you haven't been thinking about Hunchat for a long time. In our case, we've been working on this for five months, we already thought and realised what our major differences are."} I also can add that almost all new social media break an already ruling paradigm. Instagram for example... What paradigm is Hunchat breaking? And how is it doing it? \textit{"In terms of video it's changing the idea that video needs to be worked on, and thought about - right now the type of video you watch online is either an heavily produced youtube video, or a TikTok that is almost scripted, or a facebook video from brands with text and a spinning bar to catch your attention. There still isn't a place for you to grab your phone and do what you do on Instagram. I'm here talking and I publish. Done. No filters."}
Ernesto added \textit{"Let me add that there is no place, but if you look closely you'll notice that there are people, even Portuguese creators, that use it in this way. And that's where you see the opportunity. It's in this moment that you think that maybe you are not the only one wanting to use something like this. You have people using stories this way, but what happens? They will be broken into parts, because stories have a time limit. And people can only reply with text, so the intrusion between creator and consumer is not created, and there can't be a sequence. I liked a lot when some months ago  Instagram was testing a feature of stories on group chats. If you think about group stories, if you exclude the time constrain, is the kind of dynamic we want to create with Hunchat. Stories like they are messages. The dynamic is: you create a video, then someone replies with a video, then another guy sees and replies too. Now imagine that with all your friends. That's why we talk about a feed. That dynamic is not being explored. I don't know if you used that feature, but they got rid of it".} Ze: \textit{"For example the reply to on youtube is an example of people expressing the need for something as Hunchat. We are creating a place for doing that."} Then I shared some of the ideis I had for Hunchat communication, to understand if it was in accordance with their goals. Ze: \textit{"Focus more on the friends part. And avoid confronting competitors directly."} Ernesto: \textit{"I liked the idea. But using one of my famous analogies. We don't want to be the runner that's looking on the side to see how the others are doing. We want to look forward, we have a clear goal of what we want to be, what we want to provoke on users and their friends. I would focus more on that sensation. We want for you to use Hunchat like when you are on a weekend trip with your friends and everyone is surrounding a fire on a beach with a sunset, that's the vibe we are looking for. Everyone talking and sharing in a way bigger scale. I would focus more on that."} I don't know if that is enough, though. Ernesto \textit{"When you go to a supermarket and you look to a sneakers chocolate. You're not looking for it to be different from the competitors. You are only looking for the sensation you'll feel when you eat it. You are thinking on what it will offer you. They aren't worried about saying that it is sweeter that KitKat or that it contains more nuts than Cadbury. I think that it's easier to appeal to the consumer if you are talking in an emotional language - verbal and non-verbal - than trying to distinguish Hunchat. In practical terms Hunchat is not different from what already exists. I can go anywhere and send a video. For someone who does not work in social networks, that does not study social phenomena does not get the difference. You can just see by the question: What is different between Hunchat and TikTok? Everything. But as it is a video feed, people who are not used to work with this can not see the difference and that's normal. So I feel that it's way more effective for us to show what we have to offer than be here trying to explain why it's different. It can be subtle, like the example you gave. But it isn't important that for our messages to contain facts like no-limit videos, or other features. That's not why we choose that chocolate bar."} I agree. Ze: \textit{"If you check this recent app Be Real. It's basically a photo feed, nothing more. And it is an huge success. Why? Because their dynamic is that both the front and internal camera shoot at the same time, the goal is for you to show what you're doing. And that's it. That's the feature, that's the whole app. In the last month it had 250k new users."} I need to check how they communicated then. \textit{"They didn't."} I doubt that. \textit{"It's all growth hacking. They get people's contact, and inside the app they can invite, and they can make their own app their communication channel."} Ernesto added \textit{"Their communication is in the name. Be real. They are fighting the same thing we are: the influencer culture. Social media going more to the consumer side and not being a social network anymore. But they don't create content around be real. They just use the instances they have to position be real."} 

\subsection{Interview with Hunchat's Waiting List Members} \label{fwl}
All participants consented...
\subsubsection{Participant 1}
Participant 1 is male and 21.
(1) How did you get to know Hunchat? \textit{"I've known one of the founders Ze fo quite a long time now. We've worked together before on related projects. It just came up in conversation naturally. (...) As designer myself I was super interested."}
(2) Why is Hunchat different? \textit{"From what I understood and from my first impressions, it has a very sharing ideas aspect, a concept of communicating a more tangible thing than other social media. I feel other social media focus more on everyday stuff or more towards the consuming. The social aspect is way more present. I think this oned is going to be bigger with professional crowds or with people who are enthusiasts about a certain thing and want to have conversations about it. It's gonna feel a lot more as a social version of reddit. You're gonna look a topic that your interested in and you'll expand on it with people who are also enthusiasts. It's essentially more fluid version of reddit where people will hunt down topics and"}
(3) What social media do you use? \textit{"I try to be on as many of them that I need to be. I'm not on facebook because I don't need to be on facebook. I am on twitter because I feel it's the fastest one out of the bunch. I am on Instagram becaus consuminng visual media is still in that sense faster than youtube. I guess reddit - again it's a place to find information, but feels way less social than say Instagram. It's more about information than connecting with another person."} Usually the identity is anonymous. \textit{"I'd say the only one I don't use is facebook, everything else is present in my life. Just because of maintaining contact."}
(5) Do you discuss your ideas online?  \textit{"I have an interest in proactively discussing ideias. I'm always up to do things like interviews, short podcasts and even just meeting with people and talk about random things - I schedule calls. I think that's the reason why Ze and me connected. I think we have an unreleased podcasts with three episodes of us talking. I believe that ideas become better with people. So putting out ideas is really important to me. I don't know if that's because I'm present in the creative field so I have to proactively think about creativity. I don't know if it's a common thing or just a one off thing. But I feel Hunchat has a market for people who are enthusiasts about ideas, just as reddit does"}
(6) Do you communicate using video regularly?  \textit{"I haven't chosen video specifically because that out of all the paths I could take it always seems the slowest. Simply because if I'm going to do video, my options are - video calling someone which is very limiting 'cause you're just talking with one person or a group call which stays on the moment, you can't easily revisit; I could make a video and pt it on youtube, but that's a lot of work and also you're response is not the same, you are not getting a video back, it's comments of nothing. So no, video is not a common medium for me to share ideas for sure. But I see how Hunchat would make that easier and how that can emphasise a lot of things. Human require something more than just text when exploring ideas. The higher layers of communication we can have - if I can see your face, if I can hear your voice - the more they emphasise an idea. I'm more likely to listen to an idea if I like for example that person's body language. It's something that humans do subconsciously. So, I really think that if video was really easy or was as easy as posting a tweet then I would probably go for that."}
(7) Do you feel the need for better communication apps? \textit{"Yes, I do feel the need for better communication apps. I also feel the need for more than just having Instagram constantly. expand on things. I feel like it's quite a problem. For example, TikTok comes out and then Instagram puts the same feature that tiktok has in there, then Snapchat and they put that in instagram. Facebook is kind of stupid now, they're just missing a twitter section. It sounds like something they might do. Ther has to be other players in the market. We need other outlets."}
(8) Describe the product. What ate its benefits and functions? \textit{"I'll do that by imagining my experience. How I think me opening Hunchat is going to be like. It would be as often as I open reddit, or as often as I open instagram. I'm expecting to open the app and have to options: one is seeing what people I follow, or topics I care about, and how is that on my feed; and second is how good of an exploration page i think i can have - because it's really important. I will open the app and listen to new ideas about topics I care about and I either reply to those, or some sort of interaction. After sorting out through everything that I like i want to be able to find more. I think that's why I'm really looking forward to it. No matter what sort of community it will have, it will be an idea developing community. It will be a community where new ideas can be born and innovation can happen."}
(9) What need does Hunchat satisfy? \textit{"Faster innovation. Faster idea development. I don't think it's faster information delivery. Reading a tweet is a lot faster than watching a video. I think it's gonna be an enhanced experience of sharing ideas. I definitely see people opening hunchat not when they're walking, but when they are seating down. A longer experience than scrolling trhough a twitter feed on an elevator for example. Mostly because videos gonna have audio and you don't wanna be blasting audio in public. I see a future when people tweet out what they have posted on Hunchat. I think of it as a specialised version of youtube. Youtube is somewhere people put out videos, but the aspect of interaction is very small. They even removed youtube messages which was the only way to reach out to other creators. I see it as satisfying a need for creative people."}
(10) Define Hunchat with three words. \textit{"Rich, direct, and efficient. The people on ther are people who care about ideias and probably din't have that much time. They would sit down for a video that's 2-3 minutes long but they wouldm't have the patience to sit down and watch a docmentary about that thing. I see it more suceeding on the short. For ideas to reach as many people, they have to be concise."}
(12) Why did you join the waiting list?  \textit{"Because I see myself using the app."}
(13) What is different about Hunchat comparing with competition?  \textit{"Its very specific purpose. Yes, it's open in what kind of ideas you're gonna share there. It's gonna have all sort of topics. But the main concept is still having an ansychronous conversation. I think that specific goal is how it's different from something like instagram which is not really trying to do anything, it's just allowing anything to happen. Hunchat is amore proactive approach to sharing ideas or having conversations."}
 (14) What are Hunchat’s brand values?  \textit{"I'm trying to separate the project from the founders. It's very difficult. It's gonna cut the bullshit. I don't think there's gonna be any bullshit. It's gonna be clean."}
(15) Do you have meaningful discussions online?  \textit{"Yes, I think I do. But they are hard to come by. You need to do what we did - schedule it. Or even if you don't, it's always by chance. That's the genius of Hunchat. Meaningful conversation can happen in two ways: one is over a long period of correspondence or the chance of just having an around two hour frame with someone. Hunchat is gonna allow this sort of video asychornous conversation where you can have a more meaningful long term conversation or it can be as short as it has to be. It's a more 'leave it up there, and it stays' conversation. If it's not happening at the same time - No, I don't have a lot of meaningful conversation online. If I start a thread on reddit or twitter I'm more likely to forget to reply. But if someone is dedicated to open a camera and shot a videothat's when I'm gonna be likely to return the same energy."} What kind of topics would you be discussing?  \textit{"Everything. Anything but politics. (...) I'll be there using everything."}  
(19) What do you think Hunchat should communicate? \textit{"I would say the first impression is gonna count a lot. If people use the app for the fist time and it's hard to use that might ward off people. They need to focus on the first impression, for sure."}
(20) Are you concerned about privacy issues? \textit{"No. I think privacy online is gone.I see people complaining that the mic is always on or whatever - of course they do. I personally see it as a benefit. I get to see more thing I like, I look at products that I like more. I get a better experience. I think I don't really have any fears of privacy at all. The moment you're using social media you're really forsaking privacy. I would have a concern if it was more like of a one-to-one sort of conversation - if I would talk with family members like in whatsapp for example. But if it's meant to be a sharing platform, then I wouldn't see something as privacy being a major concern. Although it's something that over the past years has been blasted everywhere. People gave so much crap about facebook collecting data, when ISPs have been doing it for 20 years and noone cares about that. And are all these new companies - like all the VPN companies - that would go out of business if ISPs weren't stealing all your data. The moment TikTok came out everyone was using it and everyoneone was aware of how much data it's collecting - they are very public about it. I don't know if anyone is concerned about privacy at this point. I think it's too late to worry about it. I more interested in originality in a concept of the source of a conversation, or the fire-starter for an idea. I don't see privacy as a concern but it is a good selling point to be mentioned.}
(21) Do you feel the content you usually consume is meaningless? \textit{"Sometimes. A lot of the times actually. When I find myself dowtime opening youtube and just consuming absolute trash. I think you'd have to proactibely plan what you're going to consume - it takes effort. The mass social media right now are tryibg to shove anything into your face regardless of what it is because whatever you watch is gettin them views. They don't care of quality as much as of how much you're going to see. I consume a lot of meaningless things. That's a concern I have sometimes."}
(22) Do you feel you cannot fully express your ideas on existing social media apps? \textit{"Yes, specailly when I have an ideia about something but I don't have enough of a desire to expand on it - to reach out to people. I might have an ideia and want to put it out there, but if I just a post a story on instagram it's not gonna go anywhere; if I write a tweet is not gonna go anywhere; but if wanted to go somewhre I would have to contact people directly, I would have to call them. Which is too much of a hassle for an idea that's just not in my field for example. But I see myself having an ideia just puting out on Hunchat and never worrying about it. Be like - someone might see it, it's there."}
(23) Do you feel synchronous video calls are inconvenient? \textit{"(...) A good is example is that you tried to set this up as a focus group with different people, but it was very difficult because you had to line up a bunch of schedules. For your example I could see Hunchat being used as a research tool. You could have an asynchronised focus group - that's a concept."}
(24) Which channels seem more fit to learn about Hunchat? \textit{"Probbably through influencers. I think that's gonna be the highest medium for Hunchat. I think it would find success if it could get intellectuals - I don't think is one but someone like Gary Vee is an example. If you could get Gary Vee on Hunchat, people would be on Hunchat. People who like to talk and share ideas a lot. For example if you wanted to expand the cooking section you could get someone like Babish to hop on and do like cooking tips on Hunchat. I feel like getting people who have ideas on there is gonna get you there. Other tahn that I would say the usual ads. I do click ads. I like ads. Ads work. When I see a product I like, I click on it. There's to kind of ads: the good ones where you see the things you wanna see; and the the other ones - they're irrelevant. So advertisement and influencers would be good media to share the concept of Hunchat."}

\subsubsection{Participant 2}

(1) How did you get to know Hunchat?  \textit{"Through the founder, Ze. It was just with mouth to mouth communication and I got invested."}
(2) Why is Hunchat different? \textit{"It's interesting that in the way of communication right now we hide behind text messages. I find it really interesting that they're trying to make conversation through smartphones on the internet normal with video. Video meetings are stil fine but there's some kind of a barrier in the morden world to just reply with a video. It's like they say, it's the most natural way and I agree.}
(3) What social media do you use? \textit{"I use facebook messenger and whatsapp, but also instagram. Those are the main three. I also have twitter but I mainly use it to update myself and not to communicate myself."}
(5) Do you discuss your ideas online?\textit{"No, I don't. I'm not really active promoting things or sharing ideas. It's mainly for private conversations."}
(6) Do you communicate using video regularly?\textit{"More and more video calls. But not live I don't. I got used to, since I have spanish friends - the more latin world uses it a lot, voice messages. Here in Belgium we never use it. Before I met people from those countries I never texted throug voice messages. So I already have a transition from text to voice messages. So I believe that video could also be a transition, but the barrier is still to high so i never do it. It feels weird."}
(7) Do you feel the need for better communication apps?\textit{"Sure. I agree that trough video you understand the most. Some people can express themselves cleraly thorugh text and through voice also, but some people are just so monotone and you cannot see expressions or emotions, but with video it's just like real life - you have it all. So in that way I think communication would be the best. I think an important factor is thta it is convenient, that it is quick. If it takes me half a minute more to write a message or to reply to another I might not it, because conversations have to go super fast these days. So the moment you have to put more effort to it than it's necessary I think it gets tedious."}
(8) Describe the product. What are its benefits and functions? \textit{"Hunchat is anew social media communication platform where people can start discussions through video and then there's an whole network of people replying to the discussion, everyone can jump in, it's not like one-on-one. It's public discussions if I understand correctly. it might have changed in the past, because I was there from the start and I don't know what changed from there on. The methodology behind it is to make conversations better with video. I think they believe that video is the truest communication form. I'm guessing there's also private conversations. "}
(9) What need does Hunchat satisfy? \textit{"Being more connected to the people. Of course with the whole covid situation people feel distanced sometimes. And trough video is different than voice or text. And specially if you have international friends. When you see the person it's like almost real life. So I think that's the main benefit that it provides. It satisfies the connection with people and how you experience the conversation."}
(10) Define Hunchat with three words.\textit{"Communication, video, connecting"}
(11) What does Hunchat mean to you?\textit{"Hunchat is a new way of communicating that brings people closer."}
(12) Why did you join the waiting list?\textit{"Out of curiosity. Of course it's because I know Ze. Buy mainly because I believe in the product. I know that video is a barrier - if your hair is bad, or if you're in pijamas - I'm just curious how that barrier will go down with Hunchat. And how will I evolve as a person."}
 (14) What are Hunchat’s brand values?\textit{"I repeat myself again. They provide connection between people, a better way of communication."}
(15) Do you have meaningful discussions online?\textit{"Yes. It depends of who the recipient is."}
(16) Where or why not?\textit{"Mainly in the normal communication platforms - whatsapp and facebook messenger. the most meaningful discussion i have are there."}
(18) How do you use it? What’s good and what is missing?\textit{"I think the thing I like is how easy it is and how quick it is. For example on the facebook message app on android you have the little cricles withe peoples faces, so you don't to go to the app. I think that conveinience and how quick you can send the message is the most important thing for me in a messaging app. I don't know if any features are missing. Something I like from google chat or slack is how you can answer in threads. That's something I miss on facebook or whatsapp.}
(19) What do you think Hunchat should communicate?\textit{"I think the should focus on the main values of getting people closer and the benefits of video. Videos are gonna be slower and people don't have a lot of time, so they really need a good reason to use it. They should focus on that. Giving the reasons why they should use video."}
(20) Are you concerned about privacy issues?\textit{"I'm not really concerned with that. My philosophy is that I no longer have privacy so whatever. Of course legally they have to be correct. I work on the IT sector so I know plenty of colleagues who are really focused on privacy I don't think they would ever use that platform. So I would really think it would be a real red flag for some people to communicate through video."}
(21) Do you feel the content you usually consume is meaningless? \textit{"That's suddenly a deep question. Yes. Most of the social media - endless scrolling, them giving you things you didn't ask for and somehow you keep scrolling - is super dumb, but I do it. So yes, I think it's meaningless. Not everything is meaningless of course."}
(22) Do you feel you cannot fully express your ideas on existing social media apps?\textit{"Yes. Maybe I didn't have a specific situation. But for example when you're expressing a idea through text, it's really easy to focus on a word you wrote that wouldn't have matter in a normal live conversation. Some points are lost because people focus on one or two words, and interpet it differently. There's no entonation. I think tehre's a lot of communication errors in text."}
(23) Do you feel synchronous video calls are inconvenient?\textit{"No. Because most of the time they are planned, otherwise it's just like a phone call - if you don't want you just ignore it and let that go to voicemail. It's true that video calls are easy when you shedule it because they usualy are longer, it's not like 2 minutes. If you're really bizzy it's esaier to say when you're free and do it. But yes, sometimes they are inconvinient."}
(24) Which channels seem more fit to learn about Hunchat?\textit{"For sure it has to be close to social media and people who do use phones. On radio and television old people would be like 'video call? what is that?' - that wouldn't work. I think you have to focuson the target an those are the people who are already on instagram and other social media. I think even facebook would be too much. I think Hunchat would be mainly young people. Instagram at the moment is the younger and most modern social media platform. I'm not really a twitter user but I follow Hunchat there and I appreciate all the updates. So I guess if you want a more serious user base maybe twitter would also be a good idea."}

\subsubsection{Participant 3}

(1) How did you get to know Hunchat? \textit{"I know Ze personally from Erasmus. I saw some things he posted on Instagram and that was the first time I heard about Hunchat. So I started following them as well. So all the communication they do o that medium I see and I try to follow it. I also signed up for the beta program, but I don't have an iPhone so I can't use it yet."}
(2) Why is Hunchat different? \textit{"Talking to somebody directly instead of typing is way more personal. Because you have to think about every word when you type and it's not as real. With video you're just talking and saying what's in your mind and it's faster and feels more personal."}
(3) What social media do you use? \textit{"I use Instagram. I kind of use facebook - I don't have the app, but I'm on some groups I really need to see so I check the browser about once a day. I use reddit. And LinkedIn. I'm not on TikTok."}
(5) Do you discuss your ideas online? \textit{"I do post sometimes on LinkedIn. It might be the only platform I actually post to. It has been a long time since I posted something on Instagram. It's just more consuming I do on those platforms. On linkedIn I like to post something to build a network and stay relevant."}
(4)  \textit{"Instagram I use mostly to stay in touch with people. I enjoy it. I use reddit to browse things more targeted to what I'm doing. I follow all the subreddits about programming and things that are really relevant for me, and some funny things as well. But reddit can be a source of learning for me. Facebook I only use for the sports groups - I play soccer and everything is planned there. "}
(6) Do you communicate using video regularly? \textit{"I think it's really hard to film myself and listen to myself talk. That's a barrier I have to overcome. But when you do it often enough it's not thar big of a deal. At the moment I don't really like seeing myself talking. It's always a shock, so I try to avoid it."}
(7) Do you feel the need for better communication apps? \textit{"Actually I do. A lot of my friends complain that I'm not active enough on whatsapp when they send me a message. I'm not the person who responds really fast because I don't really like typing, so it's kind of a big barrier to answer. So I often answer two hours later and I do really feel like answering. Sometimes I even think I will be responding in an hour or so, and I only remember it on the next day. I do believe that there is a place for a new type of medium for really fast interactions."}
(8) Describe the product. What are its benefits and functions? \textit{"I'm not that up to date. What I thought it was is like a combination of snapchat and messenger - where you can communicate with videos of yourself. That was the view I had of it. But I'm not up to date with the specific features."}
(9) What need does Hunchat satisfy? \textit{"It's really easy to have a conversation with somebody, which feels more like a conversation instead of typing and having a chat. So that's what I thought they were aiming to."}
(10) Define Hunchat with three words. \textit{"Fast and Video. Those two."}
(11) What does Hunchat mean to you? \textit{"That's not an easy one. First of all because it's from a friend and I really want to support him. For sure it has a lot of potencial."}
(12) Why did you join the waiting list? \textit{"Because I wanna know how it is. Everytime I see a new product I get curious. I really want to test it. Is it as good as it promise? Does it really solve an issue. I'm really curious if it will be good."}
Do you have any idea of how you will be using it?  \textit{"Probably not that many people in Belgium will have it. So I think I won't have a big network when it starts. So to be honest I don't know how I would use it."}
(14) What are Hunchat’s brand values? \textit{"Young."}
(15) Do you have meaningful discussions online? \textit{"I often find it's valueable to talk to somebody I haven't seen in a while. So, actually I do."}
(16) Where or why not? \textit{"For me defeneitly not Instagram. For me it's more messenger and whatsapp."}
(19) What do you think Hunchat should communicate? \textit{"Since I'm not too sure what the features are. Maybe a clear overview of what it has on the first version. That might be interesting."}
(20) Are you concerned about privacy issues? \textit{"No. I really should be more concerned about it, but as long as I don't get hacked I am not concerned about privacy. When TikTok was getting bigger and bigger there were some privacy issues and that was on the back on my head and might be the reason I'm not on it. But now I'm not on it because I don't wanna waste time with it. Maybe it's not that black and white of an answer. It's not something I usually think about when downloading an app. Specially not from a smaller company."}
(21) Do you feel the content you usually consume is meaningless? \textit{"Yes. Probably most of the content is not that valuable. On facebook, there's not much on it. On instagram it's nice to see some pictures of friends but it doesn't add that much value. And on LinkedIn people create content just to get more connections. So not that much content around."}
(22) Do you feel you cannot fully express your ideas on existing social media apps? \textit{"Maybe. A friend of mine always sends me voice clips. I should be doing that more often. It's really handy, it doesn't consume so much time. But I'm not doing it so i could express myself better if I really wanted to. But at the moment I stick just with typing."}
(23) Do you feel synchronous video calls are inconvenient? \textit{"With synchronous calls it's nice for the person who calls because it gets instant feedback, but I really don't like being called."}
(24) Which channels seem more fit to learn about Hunchat? \textit{"I think Instagram is a good one. It's pretty easy for people to share. Although I do not think IG is that good for organic growth. If you post something on linkedIn it explodes. I think I would make some cross posting on Instagram and LinkedIn."}

\subsection{Survey for General Public}\label{qgp}
\textbf{1. Please indicate your gender}
( ) Female ( ) Male ( ) Other

\textbf{2. Please select the category that includes your age}
( ) 10-18   ( ) 18-24   ( ) 35-44    ( ) 45-54 ( ) 25-34   ( )  55-older

\textbf{3. Do you have meaningful conversations online?}
( ) Yes   ( ) No

(if yes then) \textbf{4. Where?}
[ ] Instagram [ ]  Twitter Omegle  [ ]  Facebook [ ]  WhatsApp  [ ]  Discord  [ ] Reddit  [ ]  YouTube  [ ]  TikTok

(if no then) \textbf{4. Why not?}
[ ] There's no channel for it  [] I don't have conversations online 

\textbf{5. Which media do you 'consume' on a regular basis?}
[] Podcasts []  YouTube [] Videos [] Subreddits  []  Tweets
[] Blogs [] PTikToks [] PEducational [] PEvents (talks, conferences...) [] Discord
[]  Twitch [] Television []  Radio [] Physical Press [] Online Press

\textbf{6. Choose one for each statement. }( Never / Rarely / Sometimes / Often / Always )
I find it difficcult to schedule online conversations. 
I communicate my ideas (goals, achievements, thoughts) online through video.
I have meaningful discussions online.
I feel the content I'm consuming is meaningless.
I feel the need for better communication apps.
I've received a video call at an inconvenient time.
I feel I cannot fully express my ideas on the existing social media apps.

\textbf{7.Are you interested in a social media app for asynchronous video conversation?}
( ) Yes ( ) No

\textbf{8. Why?} [        ...        ]

\textbf{9. What functionalities would you like on that app?}
[ ] Changing Video Speed [ ] No Time Limit  [ ] Other

\textbf{10. I would prefer the conversation to be...}
( ) Private; between me and my correspondent.  ( ) Public; on a public feed. ( ) Both.

\textbf{11. What topics would you like to discuss?}
[ ] Personal Life (achievements, lifestyle...)   
[ ] Intelectual Discussions (Politics, Philosophy, Art, Science...)     [ ] Humour (jokes, memes...)

\textbf{12. If you had to pay for exclusive features. Would you still be interested?}
( ) Yes    ( ) No

\textbf{13. Choose your level of agreement with each statement.} (Strongly Disagree /
Disagree / Neutral / Agree / Strongly Agree)
I am concerned about online privacy issues.
I am critical of the status quo.
I seek an ad-free social network.
I'm interested in sharing and discussing my ideias online.
Most social media is too superficial. I like influencer culture.
Existing social media is tearing people apart more than connecting them.
Mainstream social media is too commercial.




%Communication
\section{Competitor's Communication}\label{comp}
\end{document}
